\documentclass[english]{article}

\usepackage{charter}
\usepackage[scaled=0.87]{beramono}
\usepackage[T1]{fontenc}
\usepackage[latin9]{inputenc}
\usepackage{url}

% markup
\newcommand*\jmacro[1]{\textbf{\texttt{#1}}}
\newcommand*\jcsmacro[1]{\jmacro{\textbackslash{#1}}}
\newcommand*\joption[1]{\textbf{\texttt{#1}}}
\newcommand*\jfmacro[1]{\texttt{#1}}
\newcommand*\jfcsmacro[1]{\jfmacro{\textbackslash{#1}}}

\usepackage{babel}
\usepackage{listings}
\lstset{language={[LaTeX]TeX},
        basicstyle={\small\ttfamily},
        frame=single}
\renewcommand{\lstlistingname}{Listing}

\setcounter{tocdepth}{2}

\begin{document}

\title{The jslectureplanner package}

\author{J�rgen Spitzm�ller%
\thanks{Please report issues via \protect\url{https://github.com/jspitz/jslectureplanner}}}

\date{Version 0.7, 2015/01/04}

\maketitle

\tableofcontents

\section{Aim of this package}

I have written this package in order to facilitate (and somewhat systematize)
the planning of my university courses. The basic idea of this package
is that you write all general data of a course (date, semester, type,
course title, session titles, general information etc.) into a central
tex file (e.\,g., \emph{metadata.tex}), and then input this file
into all documents related to the course (course program, bibliography,
scripts, handouts, beamer presentations, exercises etc.), in order
to re-use the data. This is particularly helpful if you have standard
courses which are held every other semester: instead of changing the
data in every single file, you just need to change it once, in the
metadata file.

To further facilitate the planning, the package is able to calculate
the session dates of a whole semester, if the course is held weekly.
You only need to set the date of the first lecture, the package does
the rest. Moreover, the package can be used to generate a sectioned
bibliography for the course via \textsf{biblatex}.

Since the package was written for my own needs, it is somewhat tied
to my workflow. Particularly, it is tied to my context: teaching in
the Humanities at a German-speaking university. Hence, the categorization
of the package draws on this context. In particular, all strings and
formats are German. However, it is possible to customize them to some
degree, particularly to translate them to other languages. This is
described later in this manual.

This manual describes the basic user interface of the package. You
might want to have a look at the example file included in the package
in order to see how it is supposed to be used.


\section{Requirements}

The following packages are required: \textsf{etoolbox}, \textsf{advdate},
\textsf{xkeyval}, \textsf{datetime}, \textsf{calc}.


\section{General idea}

The general idea goes as follows:
\begin{itemize}
\item Set up a \emph{metadata.tex} file (or whatever name you prefer) and
insert all meta information of the course there (as described in sec.~\ref{sec:course-metadata}
and \ref{sec:lecplan}).
\item In your documents, load the package (\jcsmacro{usepackage\{jslectureplanner\}})
and then input the metadata file (\jcsmacro{input\{metadata.tex\}}).
\item Then use the macros described below (sec.~\ref{sec:general-course-info}
and \ref{sec:session-specific-info}) to retrieve the respective data.
\end{itemize}
The procedure is detailed in the following sections.


\section{Setting up course metadata\label{sec:course-metadata}}

You can set up the course metadata either via package options (if
you load the package in the metadata file; see sec.~\ref{sub:package-options})
or via specific macros (see sec.~\ref{sec:via-macros}).


\subsection{Via package options\label{sub:package-options}}


\subsubsection{Course classification\label{sec:lecture-classification}}

The course classification draws on the conventions at German universities
(or those where I have worked, for that matter). Each type is connected
to a verbose string (such as ``Seminar''), a short form (such as
``SE'') and a form that is used to denote individual sessions (as
in ``the second \emph{session} of this course'').
\begin{description}
\item [{type=<type>,}] where \joption{<type>} is one of

\begin{itemize}
\item \joption{vl}: a lecture (verbose form: ``Vorlesung''\footnote{Sec.~\ref{sec:customizing-strings}
explains how to customize these strings.}, short form: ``VL'', session form: ``Vorlesung'')
\item \joption{vo}: also a lecture\footnote{The division of ``vl'' and ``vo'' has historical (personal) reasons.}
(verbose form: ``Vorlesung'', short form: ``VO'', session form:
``Vorlesung'')
\item \joption{ps}: undergraduate seminar (verbose form: ``Proseminar'',
short form: ``PS'', session form: ``Sitzung'')
\item \joption{se}: seminar (verbose form: ``Seminar'', short form: ``SE'',
session form: ``Sitzung'')
\item \joption{ue}: practical course (verbose form: ``�bung'', short
form: ``UE'', session form: ``Sitzung'')
\item \joption{ko}: colloquium (verbose form: ``Kolloquium'', short form:
``KO'', session form: ``Sitzung'')
\item \joption{pv}: research seminar (verbose form: ``Privatissimum'',
short form: ``PV'', session form: ``Sitzung'')
\end{itemize}
\end{description}

\subsubsection{Course title}
\begin{description}
\item [{title=<title>}] Title of the course
\item [{titlesep=<separator>}] Separator between title and subtitle
\item [{subtitle=<subtitle>}] Subtitle of the course
\item [{shorttitle=<shorttitle>}] Short title of the course (for headings
etc.)
\end{description}

\subsubsection{Date and place\label{sec:date-and-place}}
\begin{description}
\item [{semester=<term>,}] where \joption{<term>} is one of

\begin{itemize}
\item \joption{fs}: spring term (``Fr�hjahrssemester'', short form: ``FS'')
\item \joption{hs}: fall term (``Herbstsemester'', short form: ``HS'')
\item \joption{ss}: summer term (``Sommersemester'', short form: ``SoSe'')
\item \joption{ws}: winter term (``Wintersemester'', short form: ``WS'')
\end{itemize}
\item [{year=<year>}]~
\item [{uni=<university>}]~
\item [{institute=<institute>}]~
\item [{startdate=<startdate>}] Date of the first session (argument\emph{
must} have the form DD/MM/YYYY). Note that this option changes the value of \jfcsmacro{today}.
See sec.~\ref{sec:resetdate} how to reset it.
\end{description}

\subsubsection{Additional information}
\begin{description}
\item [{platform=<url>}] information (e.g., URL) of the e-learning platform
\item [{officehours=<data>}] date of the office hours
\end{description}

\subsection{Via macros\label{sec:via-macros}}

The following macros provide an alternative interface to set up the course metadata described in the previous
subsection.

\subsubsection{Course classification}
\begin{itemize}
\item \jcsmacro{LecType\{<course type>\}}: Set course type;
\joption{<course type>} is one of\footnote{See above sec.~\ref{sec:lecture-classification} for a detailed
description.}

\begin{itemize}
\item \joption{vl}: lecture (\emph{Vorlesung})
\item \joption{vo}: lecture (alternative)
\item \joption{ps}: undergraduate seminar (\emph{Proseminar})
\item \joption{se}: seminar (Seminar)
\item \joption{ue}: practical course (\emph{�bung})
\item \joption{ko}: colloquium (\emph{Kolloquium})
\item \joption{pv}: research seminar (\emph{Privatissimum})
\end{itemize}
\end{itemize}

\subsubsection{Course title}
\begin{itemize}
\item \jcsmacro{LecTitle{[}<options>{]}\{<title>\}}: Set
course title; \joption{<options>} might be any set of package options
described in sec.~\ref{sub:package-options}
\item \jcsmacro{LecTitleSep\{<title separator>\}}: Set course
title-subtitle separator
\item \jcsmacro{LecSubTitle\{<subtitle>\}}: Set course subtitle
\end{itemize}

\subsubsection{Date and place}
\begin{itemize}
\item \textbackslash{}\jmacro{LecYear\{<term>\}\{<year>\}}, Set term and
year of the course; \joption{<term>} is one of\footnote{See above sec.~\ref{sec:date-and-place} for a detailed
description.}

\begin{itemize}
\item \joption{fs}: spring term (\emph{Fr�hjahrssemester})
\item \joption{hs}: fall term (\emph{Herbstsemester})
\item \joption{ss}: summer term (\emph{Sommersemester})
\item \joption{ws}: winter term (\emph{Wintersemester})
\end{itemize}
\item \jcsmacro{LecUni\{<university>\}}: Set the name of
your university
\item \jcsmacro{LecInstitute\{<institute>\}}: Set your institute
name
\item \jcsmacro{LecStartDate\{<startdate>\}}: Set date of
the first session (argument\emph{ must} have the form DD/MM/YYYY).
Note that this macro changes the value of \jfcsmacro{today}.
See sec.~\ref{sec:resetdate} how to reset it.
\end{itemize}

\subsubsection{Additional information}
\begin{itemize}
\item \jcsmacro{SetOfficeHours\{<office hours>\}}: Set date
of your office hours
\item \jcsmacro{SetPlatform\{<platform information>\}}: Set
information (e.g., URL) of the e-learning platform
\end{itemize}

\section{Setting up a course plan\label{sec:lecplan}}

In the metadata file, you can setup a plan for the course by entering
session information in chronological order. This is done via the following
macro:
\begin{itemize}
\item \jcsmacro{NewSession{[}<options>{]}\{Session title\}}


\joption{<options>} include:
\begin{itemize}
\item \joption{titlesep=<separator>}: Separator between session title and
subtitle
\item \joption{subtitle=<subtitle>}: Subtitle of the session
\item \joption{shorttitle=<shorttitle>}: Short title of the session (for
headings etc.)
\item \joption{bibsec=<keyword>}: Session keyword for the bibliography
(see below sec.~\ref{sub:bibliographies})
\item \joption{cancel=<true|false>}: Whether this session takes place.
If you specify \joption{cancel=true}, the program will mark the
session as ``canceled''. For instance,


\jcsmacro{NewSession{[}cancel=true{]}\{Dies Academicus\}}


will output something like
\begin{quote}
\textbf{15.\,05.: Dies Academicus -- keine Sitzung}
\end{quote}

(i.\,e., ``Dies Academicus -- no session'')

\end{itemize}
\item \jcsmacro{SetBreak{[}<span>{]}\{Reason\}}: Add a semester
break (e.g., holidays). This increments the counter for one week and
outputs in the program


\begin{center}
\textbf{(<span>: Reason)}
\par\end{center}


E.\,g., \jcsmacro{SetBreak{[}28.05.-{}-3.06.{]}\{Whitsun holidays\}}
expands to


\begin{center}
\textbf{(28.05.--3.06.: Whitsun holidays)}
\par\end{center}

\item \jcsmacro{SetBreaks{[}<options>{]}\{Reason\}}: Add
a semester break (e.g., holidays) that is longer than one week. This
increments the counter respectively, and outputs in the program


\begin{center}
\textbf{(<span>: Reason)}
\par\end{center}


\jmacro{<options>} include:
\begin{itemize}
\item \joption{weeks=<int>}: Number of weeks the break takes (to increment
the internal counter that is used to calculate the session dates)
\item \joption{span=<time span>}: time span (to be output on the program)
\end{itemize}
\item \jcsmacro{SetBeamerFrameBreak}: Add a frame break in
the beamer program at this position. This also increments the frame
counter which is used for subsequent program frame titles. Note that
you can insert maximally 3 breaks.
\item \jcsmacro{SetBeamerHook\{<code>\}}: Add arbitrary \LaTeX{}
code to the beamer program
\item \jcsmacro{begin\{SessionBlock\}\{<Block title>\}}\\
\ldots{} \\
\jcsmacro{end\{SessionBlock\}}:\\
Thematic block consisting of several sessions. Respective sessions
are nested inside this block.
\end{itemize}
Note that setting up a course plan changes the value of \jfcsmacro{today}.
See sec.~\ref{sec:resetdate} how to reset it.

\section{Retrieving general course information\label{sec:general-course-info}}

Once the metadata file is set up as documented above and input to
your document(s), you can retrieve the recorded as well as concatenated
information by means the following macros.


\subsection{Course data}

The following macros output general course-specific information:
\begin{itemize}
\item \jcsmacro{lectype}: Outputs the course type in short
form (e.\,g., ``SE'')
\item \jcsmacro{lectypeverb}: Outputs the course type in
verbose form (e.\,g., ``Seminar'')
\item \jcsmacro{lectypesession}: Outputs the appropriate
``session'' string (e.\,g., ``Vorlesung'' or ``Sitzung'')
\item \jcsmacro{lectitle}: Outputs the main title of the
course
\item \jcsmacro{lectitlesep}: Outputs the separator between
title and subtitle
\item \jcsmacro{lecsubtitle}: Outputs the subtitle of the
course
\item \jcsmacro{lecfulltitle}: Outputs the full title of
the course (title, separator, subtitle)
\item \jcsmacro{lecshorttitle}: Outputs the short version
of the course title
\item \jcsmacro{lecsemshort}: Outputs the short version of
the term type (e.\,g., ``WS'')
\item \jcsmacro{lecsemverb}: Outputs the verbose version
of the term type (e.\,g., ``Wintersemester'')
\item \jcsmacro{lecyear}: Outputs the year
\item \jcsmacro{lecsemester}: Outputs the short version of
the semester (e.\,g., ``WS 2014'')
\item \jcsmacro{lecsemesterverb}: Outputs the verbose version
of the semester (e.\,g., ``Wintersemester 2014'')
\item \jcsmacro{lecuniversity}: Outputs the university name
\item \jcsmacro{lecinstitute}: Outputs the institute name
\item \jcsmacro{lecplatform}: Outputs information on the
e-learning platform
\item \jcsmacro{officehours}: Outputs the office hours
\end{itemize}

\subsection{Generating lecture programs}
\label{sec:lecprog}
\begin{itemize}
\item \jcsmacro{makeprogram}: Generates a lecture program
from the metadata in the form


\begin{lstlisting}[language={[LaTeX]TeX},basicstyle={\small\ttfamily},frame=single]
\begin{labeling}{\lecprogramlistindent}
\item[<short date>] <session full title>
...
(break)
...
\item[<short date>] <session full title>
\end{labeling}
\end{lstlisting}



The \jmacro{labeling} environment is defined by the KOMA classes.
The package provides a fallback, however, if another class is used.

\item \jcsmacro{makebeamerprogram{[}<options>{]}}: Generates
a beamer-suited lecture program from the metadata in the form


\begin{lstlisting}[language={[LaTeX]TeX},basicstyle={\small\ttfamily},frame=single]
\begin[label=beamerprogram<nr>,<bfoptions>]{frame}
\frametitle{<Heading>}
\begin{description}[\lecprogramlistindent]
\item[<short date>] <session full title>
...
(break)
...
\item[<short date>] <session full title>
\end{description}
\end{frame}
\end{lstlisting}



\joption{<options>} can be one of
\begin{itemize}
\item \joption{title=<title>}: Program title (<Heading>)
\item \joption{options=<bfoptions>}: Beamer frame options
\end{itemize}

Note that multiple subsequent frames are generated if the metadata
file contains \jcsmacro{SetBeamerFrameBreak} macros.
Also note that you can refer to the frames via the automatically generated
labels \emph{beamerprogram} (for the first frame), \emph{beamerprogram2}
for the second, etc. Finally note that the number of program frames
is currently limited to 4. If you need more, you should probably rethink
your program structure.

\end{itemize}

\subsection{Generating bibliographies\label{sub:bibliographies}}

The package provides macros for easy generation of sectioned bibliographies
via \textsf{biblatex}. The macro
\begin{itemize}
\item \jcsmacro{makesessionbib}
\end{itemize}
generates code in the form

\begin{lstlisting}[basicstyle={\small\ttfamily},frame=single,moretexcs={[1]{printbibliography}}]
\section{<session full title>}
\nocite{*}
\printbibliography[keyword=<session keyword>,heading=none]
\end{lstlisting}
 for each session of the lecture which has been linked to a \textsf{biblatex}
keyword via the \joption{bibsec} option (see above sec.~\ref{sec:lecplan}).

So if you specify your session with a keyword via the \joption{bibsec}
option, and tag your Bib\TeX{} database entries with that keyword,
you will get a list of session-specific literature.


\section{Retrieving session-specific information\label{sec:session-specific-info}}

In session-specific documents, you also need information specific
to the current session. This is done as follows.


\subsection{Setting the current session}

Use \jcsmacro{ThisSession\{<nr>\}} to tell the package
which session is current. \joption{<nr>} is an integer value, e.g.
\jcsmacro{ThisSession\{3\}} for the 3rd session of
the lecture.


\subsection{Generating session-specific information for the current session}

If you have specified the current session via the macro \jcsmacro{ThisSession},
the following macros output general session-specific information:
\begin{itemize}
\item \jcsmacro{sesdate}: Date of the current session
\item \jcsmacro{sesshortdate}: Date of the current session,
short form (no year)
\item \jcsmacro{sestitle}: Main title of the current session
\item \jcsmacro{sestitlesep}: Title-subtitle separator of
the current session
\item \jcsmacro{sessubtitle}: Subtitle of the current session
\item \jcsmacro{sesfulltitle}: Full title of the current
session (title, separator, subtitle)
\item \jcsmacro{sesnr}: Number of the current session
\end{itemize}

\subsection{Generating session-specific information for an arbitrary session}

Independent of the value of \jcsmacro{ThisSession},
you can retrieve the following information for arbitrary sessions
(session number \joption{<nr>}) via the following macros:
\begin{itemize}
\item \jcsmacro{SessionTitle\{<nr>\}}: Outputs the main title
of session number \joption{<nr>}
\item \jcsmacro{SessionFullTitle\{<nr>\}}: Outputs the full
title (main title, separator, subtitle) of session number \joption{<nr>}
\item \jcsmacro{SessionShortTitle\{<nr>\}}: Outputs the short
title of session number \joption{<nr>}
\item \jcsmacro{SessionDate\{<nr>\}}: Outputs the date of
session number \joption{<nr>}
\item \jcsmacro{SessionShortDate\{<nr>\}}: Outputs the short
date (no year) of session number \joption{<nr>}
\item \jcsmacro{MakeProgramline\{<nr>\}}: Outputs a list
item line for session number \joption{<nr>} in the form


\begin{lstlisting}[language={[LaTeX]TeX},basicstyle={\small\ttfamily},frame=single]
\item[{\bfseries <short date>}] <session full title>
\end{lstlisting}


\end{itemize}


\section{Resetting the current date}
\label{sec:resetdate}

The date calculation mechanism (as provided by the \textsf{advdate} package) rewrites the value
of \jfcsmacro{today}. So after you defined a start date via the \joption{startdate} option or
the \jcsmacro{LecStartDate} macro and/or set up a course plan as described in sec.~\ref{sec:lecprog},
\jfcsmacro{today} will most likely not return today's date anymore. 
To restore the current date, use the macro \jcsmacro{ResetDate} after your course plan has been set up.

\section{Customizing the output}


\subsection{Customizing the course type strings\label{sec:customizing-strings}}

The strings of the course types can be changed via the macros:
\begin{itemize}
\item \jcsmacro{DefTypeVL\{<short form>\}\{<title>\}\{<session title>\}}
\item \jcsmacro{DefTypeVO\{<short form>\}\{<title>\}\{<session title>\}}
\item \jcsmacro{DefTypePS\{<short form>\}\{<title>\}\{<session title>\}}
\item \jcsmacro{DefTypeSE\{<short form>\}\{<title>\}\{<session title>\}}
\item \jcsmacro{DefTypeUE\{<short form>\}\{<title>\}\{<session title>\}}
\item \jcsmacro{DefTypeKO\{<short form>\}\{<title>\}\{<session title>\}}
\item \jcsmacro{DefTypePV\{<short form>\}\{<title>\}\{<session title>\}}
\end{itemize}
An English form for a lecture, thus, can be achieved by e.\,g.
\begin{itemize}
\item \jcsmacro{DefTypeVL\{LEC\}\{Lecture\}\{Session\}}
\end{itemize}
The types themselves are hardcoded. If you need further\slash{}other
types, you need to modify the package.


\subsection{Customizing the term types}

The strings of the term types can be changed via the macros:
\begin{itemize}
\item \jcsmacro{DefTypeFS\{<short form>\}\{<verbose form>\}}
\item \jcsmacro{DefTypeHS\{<short form>\}\{<verbose form>\}}
\item \jcsmacro{DefTypeSS\{<short form>\}\{<verbose form>\}}
\item \jcsmacro{DefTypeWS\{<short form>\}\{<verbose form>\}}
\end{itemize}

\subsection{Customizing the date format}

If you need to change the date format, redefine the macros \jcsmacro{lecdateformat}
(for the long format) and \jcsmacro{lecdateshortformat}
(for the short format), using the syntax provided by the \textsf{datetime}
package. By default, the two formats are defined as follows:

\begin{lstlisting}[language={[LaTeX]TeX},basicstyle={\small\ttfamily},frame=single]
\newcommand*\lecdateformat{%
    \THEDAY.\,\twodigit{\THEMONTH}.\,\THEYEAR}
\newcommand*\lecdateshortformat{%
    \THEDAY.\,\twodigit{\THEMONTH}.}
\end{lstlisting}


Use \jfmacro{\textbackslash{}renewcommand{*}} to redefine them.


\subsection{Misc. customizing}

The canceling output is ``Session title -- keine <Session>'' (``keine''
is German for ``no''). To modify or translate this, redefine the
following macro

\begin{lstlisting}[language={[LaTeX]TeX},basicstyle={\small\ttfamily},frame=single]
\newcommand*\leccancel{-- keine}
\end{lstlisting}


The default program string used in beamer frames can be changed by
redefining

\begin{lstlisting}[language={[LaTeX]TeX},basicstyle={\small\ttfamily},frame=single]
\newcommand*\lecprogram{Programm}
\end{lstlisting}


The indentation used in program lists can be changed by redefining

\begin{lstlisting}[language={[LaTeX]TeX},basicstyle={\small\ttfamily},frame=single]
\newcommand*\lecprogramlistindent{\textbf{88.\,88.}}
\end{lstlisting}


Again, use \jfcsmacro{renewcommand{*}} to redefine
those.


\section{History}

\begin{itemize}
\item 2015/01/04 (v. 0.7): Add macro for resetting \jfcsmacro{today}; minor corrections in the manual.
\item 2014/12/12 (v. 0.6\,b): Minor corrections in the manual and example files.
\item 2014/12/12 (v. 0.6): Initial release to CTAN.
\end{itemize}

\end{document}
