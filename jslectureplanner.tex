% !TeX spellcheck = en_US
%%%%%%%%%%%%%%%%%%%%%%%%%%%%%%%%%%%%%%%%%%%%%%%%%%%%%%%%%%%%%%%%%%%%%%%%%%%%%%%%%%%%%%%%%%%%%%%%%%%
%% File jslectureplanner.tex
%%
%% Manual of the jslectureplanner bundle.
%%
%% This file is part of the jslectureplanner bundle.
%%
%% Author: Juergen Spitzmueller <juergen.spitzmueller@univie.ac.at>
%%
%% This work may be distributed and/or modified under the
%% conditions of the LaTeX Project Public License, either version 1.3
%% of this license or (at your option) any later version.
%% The latest version of this license is in
%%   http://www.latex-project.org/lppl.txt
%% and version 1.3 or later is part of all distributions of LaTeX
%% version 2003/12/01 or later.
%%
%% This work has the LPPL maintenance status "maintained".
%% 
%% The Current Maintainer of this work is Juergen Spitzmueller.
%%
%% Code repository and issue tracker: https://github.com/jspitz/jslectureplanner
%%
%%%%%%%%%%%%%%%%%%%%%%%%%%%%%%%%%%%%%%%%%%%%%%%%%%%%%%%%%%%%%%%%%%%%%%%%%%%%%%%%%%%%%%%%%%%%%%%%%%%

\documentclass[english]{article}

\usepackage{metre}
\usepackage[osf]{libertine}
\usepackage[scaled=0.7]{beramono}
\usepackage[T1]{fontenc}
\usepackage[utf8]{inputenc}
\usepackage{url}

% markup
\newcommand*\jmacro[1]{\textbf{\texttt{#1}}}
\newcommand*\jcsmacro[1]{\jmacro{\textbackslash{#1}}}
\newcommand*\joption[1]{\textbf{\texttt{#1}}}
\newcommand*\jfoption[1]{\texttt{#1}}
\newcommand*\jfmacro[1]{\texttt{#1}}
\newcommand*\jfcsmacro[1]{\jfmacro{\textbackslash{#1}}}
\newcommand*\jparam[1]{\angus #1\angud}

% macros
\newcommand*\jslp{\textsf{jslectureplanner}}
\newcommand*\jsmt{\textsf{jsmembertable}}

% improve layout
\tolerance 1414
\hbadness 1414
\emergencystretch 1.5em
\hfuzz 0.3pt
\widowpenalty = 10000
\vfuzz \hfuzz
\raggedbottom

\usepackage[bookmarks=true,
bookmarksnumbered=false,
bookmarksopen=false,
breaklinks=false,
pdfborder={0 0 0},
backref=false,
colorlinks=false
]{hyperref}
\hypersetup{%
	pdftitle={The jslectureplanner manual},
	pdfauthor={Jürgen Spitzmüller},
	pdfkeywords={latex,course planning}
}

\usepackage{microtype}

\usepackage{babel}

\usepackage{listings}
\lstset{language={[LaTeX]TeX},
        basicstyle={\small\ttfamily},
        frame=single}

\setcounter{tocdepth}{2}

\begin{document}

\title{The \jslp\ bundle}

\author{Jürgen Spitzmüller%
\thanks{Please report issues via \protect\url{https://github.com/jspitz/jslectureplanner}.}}

\date{Version 1.13, 2024/03/06}

\maketitle

\begin{abstract}
\noindent The \jslp\ bundle provides a collection of packages
I have written in order to facilitate, and somewhat systematize,
the planning of my university courses and the generation of course material
(and thus, ultimately, to relieve me from some of the more boring and
time-consuming part of teaching work).
Currently, the bundle consists of the main, title-giving package \jslp\
that provides an interface to central, partly dynamically generated course
metadata as well as routines to generate programs, bibliographies etc., and one
smaller companion package: \jsmt, a package that can construct course
member and presence tables. This manual documents these packages.
\end{abstract}

\tableofcontents

\clearpage

\part{The \jslp\ package}

This part of the manual describes the basic user interface of the
\jslp\ package.
Please also refer to the example files included in the bundle in order to
see how \jslp\ is supposed to be used ``in practice''.

\section{Aim of this package}

The basic idea of the \jslp\ package is that you record
all general data of a course (date, semester, type, course title, session titles,
general information etc.) in a central tex file and then input this
file to all documents related to the course (such as the course program,
bibliography, scripts, handouts, beamer presentations, exercises etc.),
in order to re-use the recorded data.
This is particularly helpful if you have standard courses which
are held every other semester: instead of changing the data in every
single file, you just need to change it once, in the metadata file.

To further facilitate the planning, the \jslp\ package
is able to calculate the session dates of a whole semester, if the sessions
of the course follow at regular intervals.
You only need to set the date of the first session, the package does
the rest. Moreover, the package can be used to generate a sectioned
bibliography for the course via \textsf{biblatex}.

Since the package has been written for my own needs, it is somewhat tied
to my workflow. Particularly, it is tied to my context: teaching in
the Humanities at a German-speaking university. Hence, the categorization
of the package draws on this context. In particular, all default strings and
formats are German. However, it is possible to add new categories as
well as to customize and translate the strings.
This is described later in this manual.

\section{Requirements of \jslp}\label{sec:req-jslp}

The following packages are required and loaded by \jslp:
\begin{itemize}
 \setlength\itemsep{2pt}
 \item \textsf{calc}
 \item \textsf{datetime2}
 \item \textsf{etoolbox}
 \item \textsf{ifthen}
 \item \textsf{xparse}
\end{itemize}
\pagebreak

\section{General idea}

The general idea goes as follows:
\begin{itemize}
\item Set up a \emph{metadata.tex} file (or whatever name you prefer) and
record all meta information of the course in it (as described in sec.~\ref{sec:course-metadata}
and \ref{sec:lecplan}).
\item In your documents, load the package (\jcsmacro{usepackage\{jslectureplanner\}})
and/or%
\footnote{If you prefer to record the data via package options
(see sec.~\ref{sub:package-options}) instead of macros (see sec.~\ref{sec:via-macros}),
you need to load the package in the metadata file itself.}
 input the metadata file (\jcsmacro{input\{metadata.tex\}}).
\item Finally, use the macros described in sec.~\ref{sec:general-course-info}
and \ref{sec:session-specific-info} to retrieve the respective data and re-use
the recorded information in your documents.
\end{itemize}
The procedure is detailed in the following sections.


\section{Setting up course metadata\label{sec:course-metadata}}

You can set up the course metadata either via package options (if
you load the package in the metadata file; see sec.~\ref{sub:package-options})
or via specific macros (see sec.~\ref{sec:via-macros}).


\subsection{Via package options\label{sub:package-options}}

\jcsmacro{usepackage[\jparam{comma-separated options}]\{jslectureplanner\}}

\subsubsection{Course classification\label{sec:lecture-classification}}

The default course classification draws on the conventions at German-speaking
universities (or those where I have worked, for that matter). Each type is
connected to a verbose string (such as \emph{Seminar}), a short form (such as
\emph{SE}) and a form that is used to denote individual sessions (as
in ``the second \emph{session} of this course'').
\begin{description}
\item [\joption{type=\jparam{type}}] defines the course type. Predefined values are:

\begin{itemize}
\item \joption{vl}: A lecture (verbose form: \emph{Vorlesung}, short form:
      \emph{VL}, session form: \emph{Vorlesung})
\item \joption{ps}: Undergraduate seminar (verbose form: \emph{Proseminar},
      short form: \emph{PS}, session form: \emph{Sitzung})
\item \joption{se}: Seminar (verbose form: \emph{Seminar}, short form:
      \emph{SE}, session form: \emph{Sitzung})
\item \joption{ue}: Practical course (verbose form: \emph{Übung}, short
      form: \emph{UE}, session form: \emph{Sitzung})
\item \joption{ko}: Colloquium (verbose form: \emph{Kolloquium}, short form:
      \emph{KO}, session form: \emph{Sitzung})
\item \joption{pv}: Research seminar (verbose form: \emph{Privatissimum},
      short form: \emph{PV}, session form: \emph{Sitzung})
\end{itemize}

Section~\ref{sec:customizing-strings} explains how to extend this list and how
to customize the strings.
Please also cf. sec.~\ref{sec:styles} on the concept of ``styles''.
\end{description}

\subsubsection{Course title}
\begin{description}
\item [\joption{title=\jparam{title}}] Main title of the course
\item [\joption{titlesep=\jparam{separator}}] Separator between title and subtitle
\item [\joption{subtitle=\jparam{subtitle}}] Subtitle of the course
\item [\joption{shorttitle=\jparam{shorttitle}}] Short title of the course (for headings etc.)
\end{description}

\subsubsection{Date and place\label{sec:date-and-place}}
\begin{description}
\item [\joption{semester=\jparam{term}},] where \joption{\jparam{term}} is one of

\begin{itemize}
\item \joption{fs}: Spring term (\emph{Frühjahrssemester}, short form: \emph{FS})
\item \joption{hs}: Fall term (\emph{Herbstsemester}, short form: \emph{HS})
\item \joption{ss}: Summer term (\emph{Sommersemester}, short form: \emph{SoSe})
\item \joption{ws}: Winter term (\emph{Wintersemester}, short form: \emph{WS})
\end{itemize}
Sec.~\ref{sec:customizing-strings} explains how to customize and extend this list.
Also cf. sec.~\ref{sec:styles}.

\item [\joption{year=\jparam{year}}] (Start) year of the term
\item [\joption{endyear=\jparam{year}}] End year of the term if applicable (in case of terms such as \emph{Winter Term 2020/21})
\item [\joption{uni=\jparam{university}}] Your university
\item [\joption{institute=\jparam{institute}}] Your institute/department
\item [\joption{room=\jparam{room}}] The room where the course generally takes place;
      it is possible to specify diverging rooms for individual sessions
      (see sec. \ref{sec:lecplan})
\item [\joption{startdate=\jparam{startdate}}] The date of the first session (the argument
       must have the form YYYY-MM-DD or DD/MM/YYYY); if no startdate is given,
      \jfcsmacro{today} is used
\item [\joption{starttime=\jparam{starttime}}] The time the course generally starts (the argument
      must have the form HH:MM:SS)
\item [\joption{duration=\jparam{duration}}] The duration of individual course units in minutes.
      Default is 90.
\item [\joption{interval=\jparam{interval}}] The interval (in days) between two sessions;
      the preset interval is 7 (=~weekly courses)
\end{description}

\subsubsection{Additional information}
\begin{description}
\item [\joption{instructor=\jparam{name}}] (General) course instructor's name;
      it is possible to specify different or specific instructors for individual sessions
      (see sec. \ref{sec:lecplan})
\item [\joption{shortinstructor=\jparam{name}}] Short form of the course instructor's name (e.\,g., for headings)
\item [\joption{platform=\jparam{data}}] Information of the e-learning platform
      (such as the URL)
\item [\joption{officehours=\jparam{data}}] Date of your office hours
\item [\joption{officenumber=\jparam{data}}] Your office (room) number
\end{description}

\subsubsection{Other global settings}
\begin{description}
	\item[\joption{sestitlesep=\jparam{separator}}] Specifies a default separator between session title and subtitle
\end{description}

\subsection{Via macros\label{sec:via-macros}}

The following macros provide an alternative way to set up the course metadata.

\subsubsection{Course classification}
\begin{itemize}
\item \jcsmacro{LecType\{\jparam{course type}\}}: Set course type;
\joption{\jparam{course type}} is one of\footnote{See above sec.~\ref{sec:lecture-classification} for a detailed
description.}

\begin{itemize}
\item \joption{vl}: Lecture (\emph{Vorlesung})
\item \joption{ps}: Undergraduate seminar (\emph{Proseminar})
\item \joption{se}: Seminar (\emph{Seminar})
\item \joption{ue}: Practical course (\emph{Übung})
\item \joption{ko}: Colloquium (\emph{Kolloquium})
\item \joption{pv}: Research seminar (\emph{Privatissimum})
\end{itemize}
\end{itemize}

\subsubsection{Course title}
\begin{itemize}
\item \jcsmacro{LecTitle[\jparam{package options}]\{\jparam{title}\}}: Record course title and options;
\joption{\jparam{package options}} might be any set of package options described
in sec.~\ref{sub:package-options}, so this macro can actually be used to record
all metadata.
\item \jcsmacro{LecTitleSep\{\jparam{title separator}\}}: Specify a separator between course
title and subtitle (e.\,g., ``. '' or `` -- '')
\item \jcsmacro{LecSubTitle\{\jparam{subtitle}\}}: Record course subtitle
\end{itemize}

\subsubsection{Date and place}
\begin{itemize}
\item \jcsmacro{LecYear\{\jparam{term}\}\{\jparam{year}\}}, Record term and
year of the course; \joption{\jparam{term}} is one of\footnote{See above sec.~\ref{sec:date-and-place} for a detailed
description.}

\begin{itemize}
\item \joption{fs}: Spring term (\emph{Frühjahrssemester})
\item \joption{hs}: Fall term (\emph{Herbstsemester})
\item \joption{ss}: Summer term (\emph{Sommersemester})
\item \joption{ws}: Winter term (\emph{Wintersemester})
\end{itemize}
%
If the \joption{\jparam{year}} spans multiple years, always use a slash as separator (e.\,g., 2020/21)
for parsing purposes in the input.
The separator that is used in the output can be customized (see sec.~\ref{sec:custstrings}).
\item \jcsmacro{LecUni\{\jparam{university}\}}: Record the name of
your university
\item \jcsmacro{LecInstitute\{\jparam{institute}\}}: Record your institute's/department's name
\item \jcsmacro{LecRoom\{\jparam{room}\}}: Record the room where the course (generally) takes place
\item \jcsmacro{LecStartDate\{\jparam{startdate}\}}: Set the date of
       the first session (the argument must have the form YYYY-MM-DD or DD/MM/YYYY)
\item \jcsmacro{LecStartTime\{\jparam{starttime}\}}: Set the time the course generally starts
      (the argument must have the form HH:MM:SS)
\item \jcsmacro{LecDuration\{\jparam{duration}\}}: The duration of individual course units in minutes.
      Default is 90.
\item \jcsmacro{LecInterval\{\jparam{interval}\}}: Adjust the interval (in days) between two sessions.
      The preset interval is 7 (=~weekly courses).
\item \jcsmacro{SetAutoOffset[\jparam{first}]\{\jparam{n}\}\{\jparam{days}\}}:
      Automatically shift all subsequent sessions
      by \jparam{days} extra days after every \jparam{n}th session,
      counting the first session as the \jparam{first}th session for
      this purpose. This is useful if your course meets multiple times
      a week. For instance, a course meeting Tuesdays and Thursdays
      could be defined via \jcsmacro{LecInterval\{2\}} and
      \jcsmacro{SetAutoOffset\{2\}\{3\}}, a course meeting Monday,
      Wednesday, Friday using \jcsmacro{LecInterval\{2\}} and
      \jcsmacro{SetAutoOffset\{3\}\{1\}}. \jparam{n} must be a
      positive integer, \jparam{days} can also be a negative integer
      value. Use the optional \jparam{first} if the first session in
      the course is not the first session in a group. For instance, if
      your course meets Monday/Wednesday/Friday but the first day of
      term is a Wednesday, use \jcsmacro{SetAutoOffset[2]\{3\}\{1\}}.
\end{itemize}

\subsubsection{Additional information}\label{ses-add}
\begin{itemize}
\item \jcsmacro{LecInstructor[\jparam{short name}]\{\jparam{name}\}}: Record the (general) instructor's name, including an optional short form for headings etc.
\item \jcsmacro{SetOfficeHours\{\jparam{office hours}\}}: Record date of your office hours
\item \jcsmacro{SetOfficeNumber\{\jparam{office number}\}}: Record your office (room) number
\item \jcsmacro{SetPlatform\{\jparam{platform information}\}}: Record information
(such as URL) of the e-learning platform
\end{itemize}

\subsubsection{Other global settings}\label{ses-global}
\begin{itemize}
\item \jcsmacro{SessionTitleSep\{\jparam{separator}\}}: Specify a default separator between session title and subtitle (e.\,g., ``. '' or `` -- ''). This is used if no specific separator has been specified in the session entry (see sec.~\ref{sec:lecplan}); by default, no separator is defined.
\end{itemize}


\section{Setting up a course schedule}\label{sec:lecplan}

In the metadata file, you can set up a schedule for the course by entering
session information in chronological order. This is done via the following
macro:
\begin{itemize}
\item \jcsmacro{NewSession[\jparam{options}]\{Session title\}}


Valid \joption{\jparam{options}} include:
\begin{itemize}
\item \joption{draft=\jparam{true|false}}: If \joption{true}, a placeholder text (by default: ``Thema \jparam{n}'',
    see sec.~\ref{sec:listcust}) will be output if no session title is defined.
    This might be handy for planning the schedule.
\item \joption{topicnumber=\jparam{n}}: In draft mode, reset the current topic number of this session to \jparam{n}.
    All subsequent topics will be calculated from this unless reset again. This might be useful if only part of
    the schedule is still drafted.
\item \joption{titlesep=\jparam{separator}}: Separator between session title and
    subtitle (e.\,g., \verb|titlesep={.\ }|); this overrides any global separator set via \joption{sestitlesep} option
    or \jcsmacro{SessionTitleSep} macro for the current session
\item \joption{subtitle=\jparam{subtitle}}: Subtitle of the session
\item \joption{shorttitle=\jparam{shorttitle}}: Short title (for
    headings etc.)
\item \joption{instructor=\jparam{name}}: Specific instructor for this session
    (if it differs from the usual course instructor)
\item \joption{shortinstructor=\jparam{name}}: Short form of the specific instructor for this session
\item \joption{presstudents=\jparam{names}}: Students presenting in this session
    (used for student presentation lists, see below sec.~\ref{sec:preslists})
\item \joption{room=\jparam{room}}: Room for this session (if it differs
    from the usual course room)
\item \joption{starttime=\jparam{starttime}}: Start time for this session (if it differs
     from the usual start time). Input must have the form HH:MM.SS.
\item \joption{duration=\jparam{duration}}: Duration of this session in minutes (if it differs
     from the usual duration).
\item \joption{bibsec=\jparam{keyword}}: Session keyword for the bibliography
    (see below sec.~\ref{sub:bibliographies} for its use)
\item \joption{cancel=\jparam{true|false}}: Marks this session as ``canceled''.
    If you specify \joption{cancel=true}, the session will be printed
    in the program in bold type and appended by a verbal comment.
    For example,

\jcsmacro{NewSession[cancel=true]\{Dies Academicus\}}

will expand to
\begin{quote}
\textbf{15.\,05.: Dies Academicus -- keine Sitzung}
\end{quote}

(i.\,e., ``Dies Academicus -- no session'').

Furthermore, a canceled session is not counted in the session counter
that is used to retrieve session-specific information
(see below sec.~\ref{sec:session-specific-info}).
\item \joption{exam=\jparam{true|false}}: Marker for exam dates. If \joption{true},
      this session is put to the exam rather than the program list (see sec.~\ref{sec:lecprog})
      Useful if you want to list the exam dates separately. Note that exam sessions also get their
      own additional exam number, which makes them accessible independently from the number of
      preceding sessions (see sec.~\ref{sec:arbisess} on macro \jcsmacro{examsesno}).
\item \joption{extradate=\jparam{date}}: Sets a fix date for this session. The date needs
      to be inserted in ISO format (YYYY-MM-DD). Note that these sessions are excluded from
      the auto-offset set by \jcsmacro{SetAutoOffset}.
\end{itemize}
\item \jcsmacro{SetBreak[\jparam{span}]\{Break\}}: Add a semester break
(e.\,g., holidays) that affects one session. The macro increments
the internal counters respectively and expands in the program to

\begin{center}
\textbf{(Break)}
\par\end{center}

By means of the optional argument, the real time span of the break
might be given if you want to print it on the program. If you pass
an optional argument \joption{\jparam{span}}, the macro will expand to

\begin{center}
\textbf{(\jparam{span}: Break)}
\par\end{center}

So for instance, \jcsmacro{SetBreak[28.05.-{}-3.06.]\{Whitsun holidays\}}
expands to


\begin{center}
\textbf{(28.05.--3.06.: Whitsun holidays)}
\par\end{center}

\item \jcsmacro{SetBreaks[\jparam{options}]\{Break\}}: Add
a semester break that affects more than one session (e.\,g., two-week holidays).
The output in the program is identical to the \jcsmacro{SetBreak} macro,
but the counter incrementation can be adjusted to the needed time span.

Possible \jmacro{\jparam{options}} are:
\begin{itemize}
\item \joption{units=\jparam{n}}: Number of interval units the break takes. One \emph{unit}
    is the number of days as specified by the \joption{interval} option (one week by
    default\footnote{The former option \joption{weeks} is deprecated as of v. 0.9,
    since the interval can be adjusted now.}).
    This value is needed in order to increment the internal counter that is used to
    calculate the session dates. If \joption{units} is not specified, the counter is
    incremented by one.\footnote{That is: \jcsmacro{SetBreak\{Break\}} =
    \jcsmacro{SetBreaks\{Break\}} = \jcsmacro{SetBreaks[units=1]\{Break\}}.}
    \jfoption{\jparam{n}} must be a positive integer.
\item \joption{span=\jparam{span}}: Real time span (verbally output on the program)
\end{itemize}
\item \jcsmacro{SetLecOffset\{\jparam{n}\}}: Shift the date of all subsequent sessions by \jparam{n} days.
    \jparam{n} can also be negative.
\item \jcsmacro{SetBeamerFrameBreak}: Add a frame break in the beamer program at this position
    (more precisely, start a new frame; see sec.~\ref{sec:lecprog} for details).
    This also increments the frame counter which is used for subsequent program
    frame titles. Note that you can insert maximally 3 breaks.
\item \jcsmacro{SetBeamerHook\{\jparam{code}\}}: Add arbitrary LaTeX code to the
beamer program
\item \jcsmacro{begin\{SessionBlock\}\{\jparam{Block title}\}}\\
\ldots{} \\
\jcsmacro{end\{SessionBlock\}}:\\
Thematic block consisting of several sessions. Respective sessions
are nested inside this block.
\end{itemize}

\section{Retrieving general course information\label{sec:general-course-info}}

Once the metadata file is set up as documented above and input to
your document(s), you can retrieve the recorded as well as some concatenated
information by means the following macros.


\subsection{Course data}\label{sec:csdata}

The following macros output general course-specific information:
\begin{itemize}
\item \jcsmacro{lectype}: Outputs the course type in short
form (e.\,g., ``SE'')
\item \jcsmacro{lectypeverb}: Outputs the course type in
verbose form (e.\,g., ``Seminar'')
\item \jcsmacro{lectypesession}: Outputs the appropriate
``session'' string (e.\,g., ``Vorlesung'' or ``Sitzung'')
\item \jcsmacro{lectitle}: Outputs the main title of the
course
\item \jcsmacro{lectitlesep}: Outputs the separator between
title and subtitle
\item \jcsmacro{lecsubtitle}: Outputs the subtitle of the course
\item \jcsmacro{lecfulltitle}: Outputs the course's full title
    (title, separator, subtitle)
\item \jcsmacro{lecshorttitle}: Outputs the short version
of the course title; if no explicit short title is specified,
the normal title is output (you can check whether
there is a unique short title by the test
\jcsmacro{iftoggle\{uniquelecshorttitle\}\{\jparam{true}\}\{\jparam{false}\}})
\item \jcsmacro{lecsemshort}: Outputs the short version of
the term type (e.\,g., ``WS'')
\item \jcsmacro{lecsemverb}: Outputs the verbose version
of the term type (such as ``Winter\-semester'')
\item \jcsmacro{lecyear}: Outputs the (start) year
\item \jcsmacro{lecendyear}: Outputs the end year if available
\item \jcsmacro{lecendyearsep}: Outputs the separator between start and end year (slash by default)
\item \jcsmacro{lecsemester}: Outputs the short version of
the semester (e.\,g., ``WS 2014/15'')
\item \jcsmacro{lecsemesterverb}: Outputs the verbose version
of the semester (such as ``Winter\-semester 2014/15'')
\item \jcsmacro{adjsemester}: Outputs the short version of
the semester that \emph{follows} this one (e.\,g., ``SoSe 2015'')
\item \jcsmacro{adjsemesterverb}: Outputs the verbose version
of the semester that \emph{follows} this one (such as ``Sommer\-semester 2015'')
\item \jcsmacro{lecuniversity}: Outputs the university's name
\item \jcsmacro{lecinstitute}: Outputs the institute's/department's name
\item \jcsmacro{lecinstructor}: Outputs the (general) instructor's name
\item \jcsmacro{lecshortinstructor}: Outputs short form of the (general) instructor's name;
if no explicit short name is given, the normal name will be output (you can check whether
there is a unique short name by the test
\jcsmacro{iftoggle\{uniquelecshortinstructor\}\{\jparam{true}\}\{\jparam{false}\}})
\item \jcsmacro{lecroom}: Outputs the (general) course room
\item \jcsmacro{lecstarttime}: Outputs the (general) start time
\item \jcsmacro{lecduration}: Outputs the (general) duration of single units
\item \jcsmacro{lecendtime}: Outputs the (general) end time (this is calculated from the start time and duration)
\item \jcsmacro{lecslot}: Outputs the time slot of the lecture (starttime\,--\,endtime). Please refer to
      sec.~\ref{sec:custdate} for customization possibilities of the output.
\item \jcsmacro{lecplatform}: Outputs information on the
e-learning platform
\item \jcsmacro{officehours}: Outputs the office hours
\item \jcsmacro{officenumber}: Outputs the office (room) number
\end{itemize}

\subsection{Generating course programs}\label{subsec:genprog}
\label{sec:lecprog}
\begin{itemize}
\item \jcsmacro{makeprogram}: Generates a course program that, by default,
expands to the form:
\begin{lstlisting}[language={[LaTeX]TeX},basicstyle={\small\ttfamily},
                   frame=single,escapechar=\$]
\begin{labeling}{\lecprogramlistindent}
\item[$\jparam{short date}$] $\jparam{session full title}$
...
(break)
...
\item[$\jparam{short date}$]] $\jparam{session full title}$]
\end{labeling}
\end{lstlisting}

The \jmacro{labeling} environment is defined by the KOMA classes.
The package provides a fallback, however, if a different class is used.
The list type can also be customized. See sec.~\ref{sec:listcust} for details.

\item \jcsmacro{makebeamerprogram[\jparam{options}]}: Generates
a beamer-suited lecture program from the metadata in the form
\begin{lstlisting}[language={[LaTeX]TeX},basicstyle={\small\ttfamily},
                   frame=single,escapechar=\$,moretexcs={[1]{frametitle}}]
\begin{frame}<overlayarg$\jparam{n}$>[label=beamerprogram$\jparam{n}$,$\jparam{bfoptions}$]
\frametitle{$\jparam{Heading}$}
\begin{description}[\lecprogramlistindent]
\item[$\jparam{short date}$] $\jparam{session full title}$
...
(break)
...
\item[$\jparam{short date}$] $\jparam{session full title}$
\end{description}
\end{frame}
\end{lstlisting}

Again, see sec.~\ref{sec:listcust} for customization possibilities.

Note that 
\begin{itemize}
	\item multiple subsequent frames (not just slides) are generated if the metadata
	      file contains \jcsmacro{SetBeamerFrameBreak} macros (this is because beamer frame breaks
	      via \jfoption{allowframebreaks} and \jfcsmacro{framebreak} do not allow for overlays).
	\item you can refer to the frames via the automatically generated
	      labels \emph{beamerprogram} (for the first frame), \emph{beamerprogram2}
	       for the second, etc. 
	\item the number of program frames is currently limited to 4. If you need more, you should probably rethink
	      your program structure.
\end{itemize}
%
\joption{\jparam{options}} include
\begin{itemize}
\item \joption{title=\jparam{title}}: Program title (\jparam{Heading})
\item \joption{blocksonly=\jparam{true|false}}: If \joption{true}, only the session blocks are output,
      not the sessions.
\item \joption{uncover=\jparam{true|false}}: If \joption{true}, the program will be uncovered item-wise.
\item \joption{options=\jparam{bfoptions}}: Beamer frame options. Note that key-value
    beamer frame options need to be embraced, as in\\
    \jcsmacro{makebeamerprogram[options=\{shrink=10\}]}
\item \joption{overlayarg=\jparam{overlay specification}}: Beamer frame overlay specification
     (to be specified without angle brackets, e.\,g.\ \jfoption{overlayarg=1-3}).\\
     This option applies to the first program frame only. For subsequent frames,
     use \joption{overlayarg2}, \joption{overlayarg3} and \joption{overlayarg4},
     respectively.
\end{itemize}

\end{itemize}
%
If you want to separate the dates of your exams from the program, you can flag them with
the \joption{exam} option of \jcsmacro{NewSession}. Those session entries will not be included
in the output of the above two macros. However, there are the macros

\begin{itemize}
	\item \jcsmacro{makeexamprogram}. This generates a list with exam dates in the same layout
	      than \jcsmacro{makeprogram}.
	\item \jcsmacro{makebeamerexamprogram}. This generates a list with exam dates in the same layout
	      than \jcsmacro{makebeamerprogram} (and takes the same options).
\end{itemize}

\subsection{Generating bibliographies\label{sub:bibliographies}}

The package provides macros for easy generation of sectioned bibliographies
via \textsf{biblatex}. The macro
\begin{itemize}
\item \jcsmacro{makesessionbib}
\end{itemize}
generates code in the form
\begin{lstlisting}[basicstyle={\small\ttfamily},frame=single,
                   escapechar=\$,moretexcs={[1]{printbibliography}}]
\section{$\jparam{session full title}$}
\nocite{*}
\printbibliography[keyword=$\jparam{session keyword}$,heading=none]
\end{lstlisting}
 for each session of the lecture which has been linked to a \textsf{biblatex}
keyword via the \joption{bibsec} option (see above sec.~\ref{sec:lecplan}).

So if you specify your session with a keyword via the \joption{bibsec}
option, and tag your Bib\TeX{} database entries with that keyword,
you will get a list of session-specific literature.


\subsection{Generating student presentation lists}\label{sec:preslists}

The package provides macros to produce a list with the name of the students presenting
stuff, or generally being particularly involved, in specific sessions.

The names of the presenting students are assigned in the metadata to specific sessions
by means of the \joption{presstudents} option of the \jcsmacro{NewSession} command
(see above sec.~\ref{ses-add}). The macro

\begin{quote}
\jcsmacro{makepreslist}
\end{quote}
outputs a list that consists of the session dates and titles as well as the names of the
presenting students, i.\,e., it expands to code in the form:
\begin{lstlisting}[language={[LaTeX]TeX},basicstyle={\small\ttfamily},
                   escapechar=\$,frame=single]
\begin{labeling}{\lecprogramlistindent}
\item[$\jparam{short date}$] $\jparam{session full title}$\par
                    $\jparam{presenting students}$
...
\end{labeling}
\end{lstlisting}
If no presenting students have been assigned to a given session,
an em-dash (---) is output instead.
This all can be customized (please refer to sec.~\ref{sec:listcust} for details).

A starred version of the macro outputs a ``blank'' presentation list that just consists of
the session dates and titles with enough space to fill in student names manually.
This version might be useful to schedule the presentations of your students in the first
session:

\begin{quote}
\jcsmacro{makepreslist*}
\end{quote}
For beamer presentations, finally, a specific macro

\begin{quote}
\jcsmacro{makebeamerpreslist}
\end{quote}
is provided (with no starred version). The main difference is the more compact design that makes
the list fit better on a slide.

If you want to start your presentation list only from a given session number (e.\,g., since you
do not have any presentations in the first session of the course), you can use

\begin{quote}
\jcsmacro{setfirstpressession\{\jparam{session number}\}}
\end{quote}
to set the first session number to be printed on the list. Likewise, 
\begin{quote}
\jcsmacro{setlastpressession\{\jparam{session number}\}}
\end{quote}
can be used to determine the last session number to be printed on the list.


\section{Retrieving session-specific information\label{sec:session-specific-info}}

In session-specific documents, you also need information specific
to the current session. This is done as follows.


\subsection{Setting the current session}

Use \jcsmacro{ThisSession\{\jparam{n}\}} to tell the package
which session is current. \joption{\jparam{n}} is an integer value, e.\,g.
\jcsmacro{ThisSession\{3\}} for the 3rd session of the lecture.

If you are looking for exam session particularly, you can also use
\jcsmacro{examsesno\{\jparam{n}\}} with the number of the exam to get the
corresponding session number. Thus, \verb|\ThisSession{\examsesno{2}}|
will set the current session to the second exam session, notwithstanding
the number of preceding (non-exam) sessions.

\subsection{Generating session-specific information for the current session}

If you have specified the current session via the macro \jcsmacro{ThisSession},
the following macros output general session-specific information:
\begin{itemize}
\item \jcsmacro{sesdate}: Date of the current session
\item \jcsmacro{sesshortdate}: Date of the current session,
short form (no year)
\item \jcsmacro{sesdtmdate}: Date of the current session in the form of the
       \textsf{datetime2} package, without explicit style setting.
       This can be used if you want to locally set another
       date style (via \jfcsmacro{DTMsetdatestyle}, which is to be set before this macro then)
\item \jcsmacro{sesstarttime}: Start time of the current session
\item \jcsmacro{sesendtime}: End time of the current session (calculated from start time and duration).
\item \jcsmacro{seslot}: Time slot of the current session (starttime\,--\,endtime). Please refer to
sec.~\ref{sec:custdate} for customization possibilities of the output.
\item \jcsmacro{sestitle}: Main title of the current session
\item \jcsmacro{sesshorttitle}: Short title of the current session; if
no explicit short title is specified, the normal [main] title is output
(you can check whether there is a unique short title by the test
\jcsmacro{iftoggle\{uniquesesshorttitle\}\{\jparam{true}\}\{\jparam{false}\}})
\item \jcsmacro{sestitlesep}: Title-subtitle separator of
the current session
\item \jcsmacro{sessubtitle}: Subtitle of the current session
\item \jcsmacro{sesfulltitle}: Full title of the current
session (title, separator, subtitle)
\item \jcsmacro{sesblocktitle}: The title of the current session block (if available).
\item \jcsmacro{sesblocknumber}: The number of the current session block (if available).
\item \jcsmacro{sesinstructor}: Current session's instructor's name; if you want to output this name only if it differs from the general instructor (\jcsmacro{lecinstructor}), use the starred version \jcsmacro{sesinstructor*} (furthermore, you can check whether
there is a unique session instructor by the test
\jcsmacro{iftoggle\{uniquesesinstructor\}\{\jparam{true}\}\{\jparam{false}\}})
\item \jcsmacro{sesshortinstructor}: Current session's instructor's short name; if you want to output this name only if it differs from the general instructor (\jcsmacro{lecshortinstructor}), use the starred version \jcsmacro{sesshortinstructor*}; if no explicit instructor short name is given, the normal instructor name will be output (you can check whether
there is a unique short instructor [that differs from the long session instructor name] by the test
\jcsmacro{iftoggle\{uniquesesshortinstructor\}\{\jparam{true}\}\{\jparam{false}\}})
\item \jcsmacro{sespresstudents}: Students presenting in this session
\item \jcsmacro{sesnr}: Number of the current session
\item \jcsmacro{sesroom}: Room of the current session
\end{itemize}

\subsection{Generating session-specific information for adjacent sessions}

Depending on the value of \jcsmacro{ThisSession}, you can retrieve the following
information for sessions that follow or precede the current session by a specific interval unit via the following macros:
\begin{itemize}
	\item \jcsmacro{AdjSessionTitle[\jparam{offset}]}: Outputs the main title
	of the adjacent session
	\item \jcsmacro{AdjSessionFullTitle[\jparam{offset}]}: Outputs the full
	title (main title, separator, subtitle) of the adjacent session
	\item \jcsmacro{AdjSessionShortTitle[\jparam{offset}]}: Outputs the short
	title of the adjacent session; if no explicit short title is specified, the normal main title is output
	\item \jcsmacro{AdjSessionBlockTitle[\jparam{offset}]}: Outputs the session block
	title of the adjacent session (if available)
	\item \jcsmacro{AdjSessionBlockNumber[\jparam{offset}]}: Outputs the session block
	number of the adjacent session (if available)
	\item \jcsmacro{AdjSessionDate[\jparam{offset}]}: Outputs the date of the adjacent session
	\item \jcsmacro{AdjSessionShortDate[\jparam{offset}]}: Outputs the short
	date (no year) of the adjacent session
	\item \jcsmacro{AdjSessionDTMDate[\jparam{offset}]}: Date of the adjacent session in the form of the
	\textsf{datetime2} package, without explicit style setting.
	This can be used if you want to locally set another
	date style (via \jfcsmacro{DTMsetdatestyle}, which is to be set before this macro then)
	\item \jcsmacro{AdjSessionStartTime[\jparam{offset}]}: Outputs the start time of the adjacent session
	\item \jcsmacro{AdjSessionEndTime[\jparam{offset}]}: Outputs the end time of the adjacent session
	\item \jcsmacro{AdjSessionTimeSlot[\jparam{offset}]}: Outputs the time slot of the adjacent session
	      (starttime\,--\,endtime). Please refer to	sec.~\ref{sec:custdate} for customization possibilities of the output.
	\item \jcsmacro{AdjSessionInstructor[\jparam{offset}]}: Outputs the name of the instructor of the adjacent session
	\item \jcsmacro{AdjSessionInstructor*[\jparam{offset}]}: Outputs the name of the instructor of the adjacent session if it differs from the general instructor (\jcsmacro{lecinstructor})
	\item \jcsmacro{AdjSessionShortInstructor[\jparam{offset}]}: Outputs the short name (or normal name, if no short name was specified) of the instructor of the adjacent session if it differs from the general instructor (\jcsmacro{lecshortinstructor})
	\item \jcsmacro{AdjSessionShortInstructor*[\jparam{offset}]}: Outputs the short name  (or normal name, if no short name was specified) of the instructor of the adjacent session
	\item \jcsmacro{AdjSessionPresStudents[\jparam{offset}]}: Outputs the name of the students presenting in the adjacent session
	\item \jcsmacro{AdjSessionRoom[\jparam{offset}]}: Outputs the room of the adjacent session
\end{itemize}
Via the numerical \joption{\jparam{offset}} option, you can specify which session, relative to the current one, you want to output. The default is \joption{1}, i.\,e., \jcsmacro{AdjSessionTitle} without argument outputs the title of the \emph{next} session. Negative values are valid: \jcsmacro{AdjSessionTitle[-1]}, thus, outputs the title of the \emph{previous} session.

\subsection{Generating session-specific information for an arbitrary session}\label{sec:arbisess}

Independent of the value of \jcsmacro{ThisSession},
you can retrieve the following information for arbitrary sessions
(session number \joption{\jparam{n}}) via the following macros:
\begin{itemize}
\item \jcsmacro{SessionTitle\{\jparam{n}\}}: Outputs the main title
of session number \joption{\jparam{n}}
\item \jcsmacro{SessionFullTitle\{\jparam{n}\}}: Outputs the full
title (main title, separator, subtitle) of session number \joption{\jparam{n}}
\item \jcsmacro{SessionShortTitle\{\jparam{n}\}}: Outputs the short
title of session number \joption{\jparam{n}}; if
no explicit short title is specified, the normal main title is output
\item \jcsmacro{SessionBlockTitle\{\jparam{n}\}}: Outputs the session block title
of session number \joption{\jparam{n}} (if available)
\item \jcsmacro{SessionBlockNumber\{\jparam{n}\}}: Outputs the (formatted) session block number
of session number \joption{\jparam{n}} (if available)
\item \jcsmacro{SessionDate\{\jparam{n}\}}: Outputs the date of
session number \joption{\jparam{n}}
\item \jcsmacro{SessionShortDate\{\jparam{n}\}}: Outputs the short
date (no year) of session number \joption{\jparam{n}}
\item \jcsmacro{SessionDTMDate\{\jparam{n}\}}: Date of the session session number \joption{\jparam{n}}
in the form of the \textsf{datetime2} package, without explicit style setting.
This can be used if you want to locally set another
date style (via \jfcsmacro{DTMsetdatestyle}, which is to be set before this macro then)
\item \jcsmacro{SessionStartTime\{\jparam{n}\}}: Outputs the start time of
session number \joption{\jparam{n}}
\item \jcsmacro{SessionEndTime\{\jparam{n}\}}: Outputs the end time of
session number \joption{\jparam{n}}
\item \jcsmacro{SessionTimeSlot\{\jparam{n}\}}: Outputs the time slot of
session number \joption{\jparam{n}} (starttime\,--\,endtime). Please refer to
sec.~\ref{sec:custdate} for customization possibilities of the output.
\item \jcsmacro{SessionInstructor\{\jparam{n}\}}: Outputs the name of the instructor of session
number \joption{\jparam{n}}
\item \jcsmacro{SessionInstructor*\{\jparam{n}\}}: Outputs the name of the instructor of session
number \joption{\jparam{n}} if it differs from the general instructor (\jcsmacro{lecinstructor})
\item \jcsmacro{SessionShortInstructor\{\jparam{n}\}}: Outputs the short name (or normal name, if no short name was specified) of the instructor of session
number \joption{\jparam{n}}
\item \jcsmacro{SessionShortInstructor*\{\jparam{n}\}}: Outputs the short name  (or normal name, if no short name was specified) of the instructor of session
number \joption{\jparam{n}} if it differs from the general instructor (\jcsmacro{lecinstructor})
\item \jcsmacro{SessionPresStudents\{\jparam{n}\}}: Outputs the name of the students presenting
in session number \joption{\jparam{n}}
\item \jcsmacro{SessionRoom\{\jparam{n}\}}: Outputs the room of session number \joption{\jparam{n}}
\item \jcsmacro{MakeProgramline\{\jparam{n}\}}: Outputs a list
item line for session number \joption{\jparam{n}} in the form
\begin{lstlisting}[language={[LaTeX]TeX},basicstyle={\small\ttfamily},
                   escapechar=\$,frame=single]
\item[{\bfseries $\jparam{short date}$}] $\jparam{session full title}$
\end{lstlisting}

\end{itemize}
%
Note once more that if you are looking for exam session particularly, you can also use
\jcsmacro{examsesno\{\jparam{n}\}} with the number of the exam to get the
corresponding session number. For instance, \verb|\SessionDate{\examsesno{2}}|
will return the date of the second exam session, notwithstanding the global
session number of this session.


\section{Customizing the output}\label{sec:customizing-strings}

\subsection{Defining and customizing course types}

The following macro allows to define new course types and redefine existing ones:
\begin{itemize}
\item \jcsmacro{DefLecType\{\jparam{key}\}\{\jparam{short form}\}\{\jparam{title}\}\{\jparam{session title}\}}
\end{itemize}
A new type, say summer school, thus, could be defined as follows:
\begin{itemize}
\item \jcsmacro{DefLecType\{ss\}\{SS\}\{Summer School\}\{Session\}}
\end{itemize}
This could then be set via \jcsmacro{LecType\{ss\}} or the package option
\joption{type=ss}, respectively.

Likewise, you can use the macro to redefine existing styles, e.\,g. change
the ``session'' string of type \joption{vl} via
\begin{itemize}
\item \jcsmacro{DefLecType\{vl\}\{VL\}\{Vorlesung\}\{Einheit\}}
\end{itemize}
If you need to change or add types, consider the use of styles
(see sec.~\ref{sec:styles}).


\subsection{Defining and customizing term types}

The available term (semester) types can be changed and extended via the macro:
\begin{itemize}
\item \jcsmacro{DefSemType[\jparam{options}]\{\jparam{key}\}\{\jparam{short form}\}\{\jparam{verbose form}\}}
\end{itemize}
%
Options are in key-value form and include:
\begin{itemize}
	\item \joption{next=\jparam{key}}: the key of the semester type that follows this one in the academic cycle. This is
	      used for \jcsmacro{adjsemester} and \jcsmacro{adjsemesterverb}.
	\item \joption{nextny=\jparam{true|false}}: if true, the semester that follows this one is in a different year. This is
	      used for \jcsmacro{adjsemester} and \jcsmacro{adjsemesterverb}.
\end{itemize}
Thus, you could add a new term types ``spring term'' and ``autumn term'' via
\begin{lstlisting}[language={[LaTeX]TeX},basicstyle={\small\ttfamily},frame=single,moretexcs={[1]{DefSemType}}]
\DefSemType[next=at]{st}{ST}{Spring Term}
\DefSemType[next=st,nextny=true]{at}{AT}{Autumn Term}
\end{lstlisting}

\subsection{Customizing the date and time format}\label{sec:custdate}

If you need to change the date format, redefine the date styles \textsf{lecdate}
(for the long format) and \textsf{lecshortdate} (for the short format),
using the syntax provided by the \textsf{datetime2}
package. By default, the two formats are defined as follows:
\begin{lstlisting}[language={[LaTeX]TeX},basicstyle={\small\ttfamily},frame=single,moretexcs={[1]{DTMnewdatestyle}}]
\DTMnewdatestyle{lecdate}{% long date: DD.\,MM.~YYYY
\renewcommand*{\DTMdisplaydate}[4]{\DTMtwodigits{##3}.\,\DTMtwodigits{##2}.~##1}%
\renewcommand*{\DTMDisplaydate}{\DTMdisplaydate}%
}

\DTMnewdatestyle{lecshortdate}{% short date: DD.\,MM.
\renewcommand*{\DTMdisplaydate}[4]{\DTMtwodigits{##3}.\,\DTMtwodigits{##2}.}%
\renewcommand*{\DTMDisplaydate}{\DTMdisplaydate}%
}
\end{lstlisting}
\jfcsmacro{DTMtwodigits\{\#\#3\}} represents the day of the month (producing always two digits, i.\,e., \emph{01} not \emph{1}), \jfcsmacro{DTMtwodigits\{\#\#2\}} represents the month (also in two-digit form) and \jfmacro{\#\#1} represents the year. Please refer to the \textsf{datetime2} manual for details. 

If you want to redefine the styles, use \jfcsmacro{DTMrenewdatestyle} and change the definition accordingly. For instance, to get the typical American English date formats \emph{MM/DD/YYYY} and \emph{Month, DD}, redefine the formats as follows:
\begin{lstlisting}[language={[LaTeX]TeX},basicstyle={\small\ttfamily},frame=single,moretexcs={[1]{DTMrenewdatestyle}}]
\DTMrenewdatestyle{lecdate}{% long date: MM/DD/YYYY
\renewcommand*{\DTMdisplaydate}[4]{\DTMtwodigits{##2}/\DTMtwodigits{##3}/##1}%
}

\DTMrenewdatestyle{lecshortdate}{% short date: Month, DD
\renewcommand*{\DTMdisplaydate}[4]{\DTMenglishmonthname{##2}, ##3}%
}
\end{lstlisting}
Note that it is not necessary to include the \jfcsmacro{DTMDisplaydate} redefinition in the date format redefinition (since it does never change here). Note further that the macro \jfcsmacro{DTMenglishmonthname} used in the short date redefinition requires the \textsf{datetime2-english} module to be loaded (see \textsf{datetime2} manual for details).

Note, further, that if you need another date style only locally, you can set an arbitrary date style on the dtm date provided by \jcsmacro{lectdtmdate} and friends, i.\,e., \verb|\DTMsetdatestyle{mydatestyle}\lecdtmdate|.

Note, finally, that the date redefinition syntax used up to \jslp\ 1.2 is no longer supported, since it relied on the \textsf{datetime} package, which is deprecated (and superseded by \textsf{datetime2}). The \jslp\ package will issue a warning if you try to use the old syntax.

As for the time span, this is defined in the macro \jcsmacro{jstimeslot}, which is predefined as follows and can be adjusted to your needs:
\begin{lstlisting}[language={[LaTeX]TeX},basicstyle={\small\ttfamily},frame=single,moretexcs={[1]{jstimeslot}}]
\newcommand*\jstimeslot[2]{#1\,--\,#2}
\end{lstlisting}
The time format itself can be customized by the means of \textsf{datetime2} (timestyle). Please refer to the package's manual for details.


\subsection{Customizing program and presentation list appearance}\label{sec:listcust}

\subsubsection{List layout}
The course program generated by \jcsmacro{makeprogram} is typeset using
a \textit{labeling} list, i.\,e., a description list
with a customizable indentation (see sec.~\ref{subsec:genprog}).
Since \jslp\ uses customizable environments, this can be easily changed.
Simply redefine the following environment (\jcsmacro{lecprogramlistindent}
holds a string that represents the indentation width; see below):

\begin{lstlisting}[language={[LaTeX]TeX},basicstyle={\small\ttfamily},
frame=single,moretexcs={[1]{lecprogramlistindent}}]
\newenvironment{ProgramList}{\begin{labeling}{\lecprogramlistindent}}{\end{labeling}}
\end{lstlisting}
%
This will also change the list of the exam sessions generated by \jcsmacro{makeexamprogram}
and the student presentation lists generated by \jcsmacro{makepreslist}.
If you want these lists to look different from the normal program list, redefine:

\begin{lstlisting}[language={[LaTeX]TeX},basicstyle={\small\ttfamily},frame=single]
\newenvironment{ProgramListExam}{\begin{ProgramList}}{\end{ProgramList}}
\end{lstlisting}
%
\begin{lstlisting}[language={[LaTeX]TeX},basicstyle={\small\ttfamily},frame=single]
\newenvironment{PresList}{\begin{ProgramList}}{\end{ProgramList}}
\end{lstlisting}
%
For beamer programs, exam and presentation lists, a description environment is used instead,
and the respective commands are
\begin{lstlisting}[language={[LaTeX]TeX},basicstyle={\small\ttfamily},
frame=single,moretexcs={[1]{lecprogramlistindent}}]
\newenvironment{BeamerProgramList}{\begin{description}[\lecprogramlistindent]}%
                                  {\end{description}}
\end{lstlisting}
%
\begin{lstlisting}[language={[LaTeX]TeX},basicstyle={\small\ttfamily},
frame=single,moretexcs={[1]{lecprogramlistindent}}]
\newenvironment{BeamerPresList}{\begin{BeamerProgramList}}{\end{BeamerProgramList}}
\end{lstlisting}
%
\begin{lstlisting}[language={[LaTeX]TeX},basicstyle={\small\ttfamily},
frame=single,moretexcs={[1]{lecprogramlistindent}}]
\newenvironment{BeamerProgramListExam}{\begin{BeamerProgramList}}{\end{BeamerProgramList}}
\end{lstlisting}
%
The individual lines of the program are defined in the following macro that can be redefined
(\texttt{\#1} represents the date, \texttt{\#2} the session title, \texttt{\#3} the session title--subtitle separator,
and \texttt{\#4} the session subtitle; for the used text formatting macros, see sec.~\ref{subsec:textform}):
\begin{lstlisting}[language={[LaTeX]TeX},basicstyle={\small\ttfamily},
frame=single,moretexcs={[3]{ProgramListItem,programdateformat,sestitleformat}}]
\newcommand*{\ProgramListItem}[4]{\item[\programdateformat{#1}] \sestitleformat{#2#3#4}}
\end{lstlisting}
%
This also redefines the presentation list items. 
If you want these lists to look different from the normal program list, redefine:
\begin{lstlisting}[language={[LaTeX]TeX},basicstyle={\small\ttfamily},
frame=single,moretexcs={[2]{PresListItem,ProgramListItem}}]
\newcommand*\PresListItem[4]{%
    \ProgramListItem{#1}{#2}{#3}{#4}%
}
\end{lstlisting}
%
Analogous macros for canceled session items, exam items and presentation list items are:
\begin{lstlisting}[language={[LaTeX]TeX},basicstyle={\small\ttfamily},
frame=single,moretexcs={[3]{ProgramListCancelItem,programdateformat,cansestitleformat}}]
\newcommand*{\ProgramListCancelItem}[4]{\item[\programdateformat{#1}]
                                             \cansestitleformat{#2#3#4\leccancel}}
\end{lstlisting}
\begin{lstlisting}[language={[LaTeX]TeX},basicstyle={\small\ttfamily},
frame=single,moretexcs={[3]{ProgramListExamItem,programdateformat,exsestitleformat}}]
\newcommand*{\ProgramListExamItem}[4]{\item[\programdateformat{#1}]
                                           \exsestitleformat{#2#3#4}}
\end{lstlisting}
%
For the beamer program, the respective macros are:
\begin{lstlisting}[language={[LaTeX]TeX},basicstyle={\small\ttfamily},
frame=single,moretexcs={[3]{BeamerProgramListItem,programdateformat,sestitleformat}}]
\newcommand*{\BeamerProgramListItem}[4]{\item[\programdateformat{#1}]
                                              \sestitleformat{#2#3#4}}
\end{lstlisting}
\begin{lstlisting}[language={[LaTeX]TeX},basicstyle={\small\ttfamily},
frame=single,moretexcs={[2]{BeamerPresListItem,BeamerProgramListItem}}]
\newcommand*\BeamerPresListItem[4]{%
    \BeamerProgramListItem{#1}{#2}{#3}{#4}%
}
\end{lstlisting}
\begin{lstlisting}[language={[LaTeX]TeX},basicstyle={\small\ttfamily},
frame=single,moretexcs={[3]{BeamerProgramListCancelItem,programdateformat,cansestitleformat}}]
\newcommand*{\BeamerProgramListCancelItem}[4]{\item[\programdateformat{#1}]
                                              \cansestitleformat{#2#3#4\leccancel}}
\end{lstlisting}
\begin{lstlisting}[language={[LaTeX]TeX},basicstyle={\small\ttfamily},
frame=single,moretexcs={[3]{BeamerProgramListExamItem,programdateformat,exsestitleformat}}]
\newcommand*{\BeamerProgramListExamItem}[4]{\item[\programdateformat{#1}]
                                            \exsestitleformat{#2#3#4}}
\end{lstlisting}
%
The program block titles are set in the embedding list via the following macro:
\begin{lstlisting}[language={[LaTeX]TeX},basicstyle={\small\ttfamily},
frame=single,moretexcs={[3]{ProgramBlockItem,blocknumberformat,blocktitleformat}}]
\newcommand*{\ProgramBlockItem}[2]{\item[\blocknumberformat{#1}.] \blocktitleformat{#2}}
\end{lstlisting}
%
For beamer, the following is used with \joption{blocksonly=false}:
\begin{lstlisting}[language={[LaTeX]TeX},basicstyle={\small\ttfamily},
frame=single,moretexcs={[3]{BeamerProgramBlockItem,blocknumberformat,blocktitleformat}}]
\newcommand*{\BeamerProgramBlockItem}[2]{\item[\blocknumberformat{#1}.]
                                              \blocktitleformat{#2}}
\end{lstlisting}
%
and the following is used with \joption{blocksonly=true}:
\begin{lstlisting}[language={[LaTeX]TeX},basicstyle={\small\ttfamily},
frame=single,moretexcs={[1]{BeamerProgramBlockBlocksOnlyItem}}]
\newcommand*{\BeamerProgramBlockBlocksOnlyItem}[2]{\item[#1.] #2}
\end{lstlisting}
%
The indentation used in program lists can be changed by redefining
\begin{lstlisting}[language={[LaTeX]TeX},basicstyle={\small\ttfamily},frame=single,
moretexcs={[2]{lecprogramlistindent,programdateformat}}]
\newcommand*\lecprogramlistindent{\programdateformat{88.\,88.}}
\end{lstlisting}
%
The vertical space that separates lines in blank presentation lists can be changed
by redefining
\begin{lstlisting}[language={[LaTeX]TeX},basicstyle={\small\ttfamily},frame=single,
moretexcs={[2]{blankpreslistvspace,setlength}}]
\setlength\blankpreslistvspace{2\baselineskip}
\end{lstlisting}
%
The separator between session title and presenting students in presentation lists
can be adjusted by changing
\begin{lstlisting}[language={[LaTeX]TeX},basicstyle={\small\ttfamily},frame=single,
moretexcs={[1]{presseparator}}]
\newcommand*\presseparator{\par}
\end{lstlisting}
%
For beamer presentation lists, the adaptation can be done by changing
\begin{lstlisting}[language={[LaTeX]TeX},basicstyle={\small\ttfamily},frame=single,
moretexcs={[1]{beamerpresseparator}}]
\newcommand*\beamerpresseparator{:\ }
\end{lstlisting}


\subsubsection{Text formatting}\label{subsec:textform}

If you only want to change the formatting of lines, you do not need to redefine the whole macros
describe above. You can simply adjust the respective markup macros, as described in what follows.

The dates in the program are bold by default. You can change this globally by redefining
\begin{lstlisting}[language={[LaTeX]TeX},basicstyle={\small\ttfamily},frame=single,
moretexcs={[1]{programdateformat}}]
\newcommand*\programdateformat[1]{\textbf{#1}}
\end{lstlisting}
%
Cancelled session titles in the program are bold by default as well. You can change this globally by redefining
\begin{lstlisting}[language={[LaTeX]TeX},basicstyle={\small\ttfamily},frame=single,
frame=single,moretexcs={[1]{cansestitleformat}}]
\newcommand*\cansestitleformat[1]{\textbf{#1}}
\end{lstlisting}
%
Normal session titles, on the other hand, are by default not emphasized at all. You can also change this
globally by redefining
\begin{lstlisting}[language={[LaTeX]TeX},basicstyle={\small\ttfamily},frame=single,
frame=single,moretexcs={[1]{sestitleformat}}]
\newcommand*\sestitleformat[1]{#1}
\end{lstlisting}
%
Exam session titles are copies of the above. You can also change this
globally by redefining
\begin{lstlisting}[language={[LaTeX]TeX},basicstyle={\small\ttfamily},frame=single,
frame=single,moretexcs={[2]{exsestitleformat,sestitleformat}}]
\newcommand*\exsestitleformat[1]{\sestitleformat{#1}}
\end{lstlisting}
%
Session block titles and session block numbers can be redefined by changing
\begin{lstlisting}[language={[LaTeX]TeX},basicstyle={\small\ttfamily},frame=single,
frame=single,moretexcs={[1]{blocktitleformat}}]
\newcommand*\blocktitleformat[1]{\textbf{#1}}
\end{lstlisting}
\begin{lstlisting}[language={[LaTeX]TeX},basicstyle={\small\ttfamily},frame=single,
frame=single,moretexcs={[1]{blocknumberformat}}]
\newcommand*\blocknumberformat[1]{\textbf{#1}}
\end{lstlisting}
%
If you want to change the line in which semester breaks are output, redefine
\begin{lstlisting}[language={[LaTeX]TeX},basicstyle={\small\ttfamily},frame=single,
frame=single,moretexcs={[1]{breakevent}}]
\newcommand*{\ProgramListBreak}[2][\empty]%
{%
  \begin{center}
    (\breakevent{#1}{: }{#2})
  \end{center}%
}
\end{lstlisting}
%
\jcsmacro{breakevent} thereby checks whether a date range has been given (\texttt{\#1}),
and if so, it is output followed by the separator given as second argument and the break
title (\texttt{\#2}). If not, only the break title (\texttt{\#2}) is output.

\subsubsection{Strings}\label{sec:custstrings}

For canceled sessions, the program outputs
``\jparam{Session title} -- keine \jparam{Session}'' (``keine''
is German for ``no''). To modify or translate this, redefine the
following macro
\begin{lstlisting}[language={[LaTeX]TeX},basicstyle={\small\ttfamily},frame=single,
                   moretexcs={[2]{leccancel,lectypesession}}]
\newcommand*\leccancel{\ -- keine \lectypesession}
\end{lstlisting}
The default program string used in beamer frames can be changed by
redefining
\begin{lstlisting}[language={[LaTeX]TeX},basicstyle={\small\ttfamily},frame=single,
                             moretexcs={[1]{lecprogram}}]
\newcommand*\lecprogram{Programm}
\end{lstlisting}
The placeholder text for session without title with \joption{draft=true}
(see sec.~\ref{sec:lecplan}) is ``Thema \jparam{n}'' (``Thema'' is German for ``topic'').
It can be changed by redefinition of this macro (\jcsmacro{thesestopic} outputs
the value of the internal topic counter).
\begin{lstlisting}[language={[LaTeX]TeX},basicstyle={\small\ttfamily},frame=single,
                   moretexcs={[2]{sestopic,thesestopic}}]
\newcommand*\sestopic{Thema~\thesestopic}
\end{lstlisting}
%
The placeholder for presentation sessions with no students assigned (by default
an em-dash), can be redefined by changing
\begin{lstlisting}[language={[LaTeX]TeX},basicstyle={\small\ttfamily},frame=single,
moretexcs={[1]{emptypressession}}]
\newcommand*\emptypressession{---}
\end{lstlisting}
%
The separator between start and end year of a term can be redefined by changing
\begin{lstlisting}[language={[LaTeX]TeX},basicstyle={\small\ttfamily},frame=single,
moretexcs={[1]{lecendyearsep}}]
\newcommand*\lecendyearsep{/}
\end{lstlisting}

\subsection{Using styles}\label{sec:styles}

The most elegant way to do the customizing is to use \emph{lecture planner style} (*.lps) files.
These files can be loaded via the \joption{style=\jparam{style}} package option, where \jparam{style}
is the file name without *.lps extension.

Currently, the package ships three styles:
\begin{enumerate}
  \item \joption{german-default} simply includes the default strings.
     This style is meant to be used as a model for new styles.
  \item \joption{english} is a proof-of-concept localization to English.
     This style can be used as a basis for English styles. The English examples
     included in the \jslp\ bundle use this style.
  \item \joption{cologne} provides some specifics current at the University of Cologne
     (\emph{Universität zu Köln}):
     \begin{itemize}
      \item Additional course types: \joption{ak} (AK, \emph{Arbeitskurs}),
      \joption{as} (AS, \emph{Aufbauseminar}), \joption{es} (ES, \emph{Einführungsseminar}),
      \joption{evl} (EVL, \emph{Einführungsvorlesung}), \joption{hs} (HS, \emph{Hauptseminar}).
     \end{itemize}
     
\end{enumerate}
If you provide me with other style files, in particular
translations to other languages than German (but also other university conventions), I consider
them for inclusion to the package.

\clearpage

\part{Companion packages}

\section{The \jsmt\ package}

This section of the manual describes the user interface of the \jsmt\
package. Please also refer to the example files included in the bundle.

\subsection{Aim of this package}

This package provides a way to easily generate tables to track the members of a
university course (a table where you, or your students, can fill in names,
student IDs and e-mail adresses) and a table to record their presence,
should this be a requisite at your institute.

The package somewhat integrates with \jslp\ and can use the
metadata and styles of that package where necessary (see part~I of this
manual for details on metadata and styles).

Since the package was written for my own needs, the table design might not fit yours.
However, with some basic LaTeX knowledge, it should not be too hard to adapt it.
Like for \jslp, the default strings are German.
However, it is possible to customize and translate them.
This is described later in this manual.


\subsection{Requirements of \jsmt}\label{sec:req-jsmt} 

The following packages are required and loaded by \jsmt: 
\begin{itemize}
 \setlength\itemsep{2pt}
 \item \textsf{datatool} (only if the option \joption{datafile} is used)
 \item \textsf{ifthen}
 \item \textsf{calc}
 \item \textsf{longtable}
 \item \textsf{hhline}
\end{itemize}
\jsmt\ can be used independently of
\jslp\ to some degree. The latter package is only required for specific
features (see below) and must be loaded explicitly in that case.

\subsection{Loading the package}

The package is loaded as usual:
\begin{quote}
\jcsmacro{usepackage[\jparam{options}]\{jsmembertable\}}
\end{quote}
Currently, the following package options are available:
\begin{itemize}
 \item \joption{datafile=\jparam{csv-file}}: Use external student data file (see sec.~\ref{sec:ext-data}).
 \item \joption{datakeys=\jparam{keyorder}}: Specify order of items in CSV data file (see sec.~\ref{sec:ext-data}).
\end{itemize}


\subsection{Loading external data}\label{sec:ext-data} 

Both course member and presence tables can be filled with external student data. To this end,
the package option \joption{datafile=\jparam{csv-file}} is provided. If it is used, the package tries
to access a comma-separated data file. Note that the external file must have the extension \emph{.csv},
and the extension must be omitted in the option, so
\begin{quote}
 \jcsmacro{usepackage[datafile=mystudents]\{jsmembertable\}}
\end{quote}
loads the file \texttt{mystudents.csv}.

It is assumed that the external CSV file conforms to the following structure:
\begin{lstlisting}[language={},basicstyle={\small\ttfamily},frame=single]
Prename1, Surname1, StudentID1, Email1
Prename2, Surname2, StudentID2, Email2
...
\end{lstlisting}
The order if items can be adjusted via the \joption{datakeys} package option.
The four items, \texttt{pname} (prename), \texttt{sname} (surname), \texttt{mid} (student ID) and \texttt{email}, can be freely re-ordered by that option. Note however, that all four items must be used. Thus, to change the order to \emph{Surname, Prename, Email, StudentID}, use:
\begin{quote}
	\jcsmacro{usepackage[datafile=mystudents,\\
	\hphantom{\textbackslash usepackage[}datakeys=\{sname,pname,email,mid\}]\{jsmembertable\}}
\end{quote}
Note, finally, that the CSV file should have the same encoding than the tex file.

\subsection{Generating a course member table}

In order to generate a course member table, simply insert

\begin{quote}
\jcsmacro{makemembertable\{\jparam{number of blank rows}\}}
\end{quote}
If an external data file has been loaded, \joption{\jparam{number of blank rows}} can either be empty or a positive
integer value. If no external data file has been loaded, \joption{\jparam{number of blank rows}} \emph{must} be a
positive integer value.

If no external data file has been loaded, the command generates a table with \joption{\jparam{number of blank rows}}
blank rows, to which your students can fill in their name, student ID and e-mail address.

If an external data file has been loaded, an appropriate number of rows is generated and the
respective student data is filled in. Additionally, \joption{\jparam{number of blank rows}} blank rows are
\emph{appended} after the filled-in rows (for the late birds that have not registered themselves yet).
If the command's argument is empty or \joption{0}, no extra rows are appended.

The number of columns of the member table is always 3, tracking \emph{student id},
\emph{name} and \emph{e-mail address}.
If you want to have more or less columns, you need to redefine the
\jcsmacro{makemembertable} command.
The header strings, though, can easily be changed by redefining the following three
macros (using \jfcsmacro{renewcommand*}), for instance like this:
\begin{lstlisting}[language={[LaTeX]TeX},basicstyle={\small\ttfamily},frame=single]
% Student Name
\renewcommand*\jsmnameheader{\textbf{Student name}}
% Student ID
\renewcommand*\jsmidheader{\textbf{Student ID}}
% Email
\renewcommand*\jsemailheader{\textbf{Email address}}
\end{lstlisting}
If you use the \jslp\ package, you can use the \jslp\
style files in order to store those redefinitions globally.
See the \texttt{german-default.lps} style file for a model.\footnote{%
Note that within styles, you must use \jfcsmacro{newcommand*} instead of
\jfcsmacro{renewcommand*}, since the styles are loaded before the \jsmt\ package.}
Of course, you need to load \jslp\ with the respective \joption{style} option then
(\emph{before} \jsmt).
See above section~\ref{sec:styles} for details on styles.

\subsection{Generating a presence table}

Presence tables consist of rows with your students' names and columns for each session
of your course where each student can sign and mark her/his presence in the respective
session. As with the member tables, the package can fill in student data (the student
name in this case) from an external data file if the \joption{datafile} package option
is used.

The package provides two kinds of presence tables: A presence table with real session dates
and a simpler one just with session numbers (``Session~1'' etc.). The first variant requires
the \jslp\ package, since it uses the course metadata in order to generate a matching number
of columns with the respective session date in the column headers. The second one also works
without \jslp\, since the session numbers can be simply derived from an internal counter and
you can manually set the total number of sessions.

Note that if you use \jslp, you must load the \jslp\ package and also input the metadata file
\emph{before} \jsmt.

The first presence table variant (with real session dates) is produced with the following macro:

\begin{quote}
\jcsmacro{makeprestable\{\jparam{number of blank rows}\}}
\end{quote}
As with the member table, an appropriate number of rows, pre-filled with student data, is
generated if an external data file has been loaded. Then, \joption{\jparam{number of blank rows}}
blank rows are \emph{appended} after the filled-in rows (if \joption{\jparam{number of blank rows}}
is not empty or \joption{0}). If no external data file has been loaded, simply a table with
\joption{\jparam{number of blank rows}} blank rows is constructed.

The number of columns/sessions is automatically calculated from the metadata. If you want a different
number, you can use the optional argument of the command to override this calculation:

\begin{quote}
\jcsmacro{makeprestable[\jparam{number of sessions}]\{\jparam{number of blank rows}\}}
\end{quote}
The second kind of presence table (with session numbers instead of session dates) is provided
via the starred version of the command:

\begin{quote}
\jcsmacro{makeprestable*[\jparam{number of sessions}]\{\jparam{number of blank rows}\}}
\end{quote}
Like with the unstarred version, if the \jslp\
metadata is used/available, the number of columns/sessions is calculated from the metadata.
Again, the optional argument can be used to override this calculation.
If the \jslp\ metadata is not used/available, and no session number is given via the optional
argument, the table will use a preset number of sessions, namely 15. The starred version uses,
like the unstarred variant, the external student data if the \joption{datafile} option has been
used.

For courses of 2--16 sessions, the package tries to produce reasonably sized tables that
fit the page width (more than 16 sessions are currently not supported).
From 7 sessions upwards, two separate tables, each for half a term,
are produced. In general, it is assumed that you use landscape page format for presence tables.
The tables will be too wide for portrait page format. Also, the tables have been optimized for A4
page size.

The header strings of the presence tables can be adjusted by redefining the following
macros (using \jfcsmacro{renewcommand*}), for instance like this:
\begin{lstlisting}[language={[LaTeX]TeX},basicstyle={\small\ttfamily},frame=single]
% Student Name
\renewcommand*\jsmnameheader{\textbf{Student name}}
% "Session" string for the starred macro version
\renewcommand*\jsmsession{Session}}
% Session header for the starred macro version
% (\thesession outputs the counter value)
\renewcommand*\jsmsessionheader{\textbf{\jsmsession\ \thesession}}
% The advice for students to sign
\renewcommand*\jssigheader{Students' signature}
\end{lstlisting}
Again, if you use \jslp, you can use the \jslp\
style files in order to store those redefinitions globally.
See the \texttt{german-default.lps} and \texttt{english.lps} style files for a model.\footnote{%
Note that within styles, you must use \jfcsmacro{newcommand*} instead of
\jfcsmacro{renewcommand*}, since the styles are loaded before the \jsmt\ package.}
Of course, you need to load \jslp\ with the respective \joption{style} option then
(\emph{before} \jsmt).
See above section~\ref{sec:styles} for details on styles.

Note, furthermore, that the ``Session'' string (value of \jcsmacro{jsmsession}) is by default
linked to \jslp's \jcsmacro{lectypesession} value if 
\jslp\ is loaded before \jsmt. So if \jcsmacro{lectypesession} fits, you do not have to redefine
\jcsmacro{jsmsession}.


\clearpage

\appendix

\part{Appendix}

\section{Acknowledgements}

I could not have written the packages of the \jslp\ bundle without the expertise of the incredible \LaTeX\
community. The packages heavily rely on functions and features
provided by other packages (which are listed in sec.~\ref{sec:req-jslp} and \ref{sec:req-jsmt}).
I am deeply indebted to the authors of those packages.
Furthermore, \emph{stackexchange} proved to be a great source of knowledge which
provided me with solutions for specific tasks:
\jsmt\ heavily draws on code for dynamic table generation that has been
provided by Herbert Voss (\url{http://tex.stackexchange.com/a/7594}); \jslp\ and \jsmt\
use code to test for \jfcsmacro{romannumeral} arguments which has been taken from
a \emph{stackexchange} post by Enrico Gregorio (\url{http://tex.stackexchange.com/a/50131}).
The time span calculation uses code provided by Christian Hupfer at
\url{https://tex.stackexchange.com/a/473552/19291}.
Thank you, Herbert, Enrico and Christian!
Richard Zach provided a fix and enhancement to \jcsmacro{SetAutoOffset} and code to customize
the program list. Many thanks for this.
Finally, Dominik Waßenhoven tested the package and provided
me with multiple suggestions, which effectively manifested in major new features such as
the \jslp\ style files. Thanks, Dominik.

\section{Release History}

\begin{itemize}
\item 2024/03/06 (v. 1.13):
\begin{itemize}
	\item Use legacy keyvals rather than \textsf{xkeyval}.
	\item Fix row spacing in prestables.
\end{itemize}
\item 2020/12/08 (v. 1.12):
	\begin{itemize}
		\item Correctly parse staryear/endyear.
	\end{itemize}
\item 2020/05/11 (v. 1.11):
	\begin{itemize}
		\item Allow for customization of program and presentation lists (see sec.~\ref{sec:listcust}).
		\item Correctly set first session in presentation list.
	\end{itemize}
\item 2020/04/30 (v. 1.10):
	\begin{itemize}
		\item Do not increment session topic counter for cancelled sessions and breaks.
		\item Add session option \joption{topicnumber} to reset topic number in draft mode.
		\item Recode this manual in utf8.
	\end{itemize}
\item 2020/02/04 (v. 1.9):
	\begin{itemize}
		\item Allow \jcsmacro{SetBeamerFrameBreak} within \joption{SessionBlock}.
		\item Fix \jcsmacro{SetAutoOffset} at first session and add optional argument
		      to alter first session (fix contributed by Richard Zach).
	\end{itemize}
\item 2019/02/06 (v. 1.8):
	\begin{itemize}
		\item Add support for displaying the next semester (macros \jcsmacro{adjsemester} and
		      \jcsmacro{adjsemesterverb}).
		\item Add support for lecture and session start time, duration and thus also end time
		      calculation.
		\item Add unstyled date formats which allow preceding \jfcsmacro{DTMsetdatestyle}
		      (the short and long date formats would override such settings).
		\item Add \joption{extradate} option to \jcsmacro{NewSession} to support fix dates.
		      See sec.~\ref{sec:lecplan}.
		\item Add possibility to enter exam dates that are not added to the standard program
		      (option \joption{exam} of \jcsmacro{NewSession}); see sec.~\ref{sec:lecplan}.
		      Also add macros to display them separately (\jcsmacro{makeexamprogram},
		      \jcsmacro{makebeamerexamprogram}; see sec.~\ref{sec:lecprog}) and to
		      access their session number (\jcsmacro{examsesno}; see sec.~\ref{sec:arbisess}).
		\item Switch input format of \jcsmacro{LecStartDate} to ISO (YYYY-MM-DD). The old format
		      (DD/MM/YYYY) is still supported for backwards compatibility.
		\item Add macros to receive session block numbers (\jcsmacro{sesblocknumber},\\
		      \jcsmacro{SessionBlockNumber\{\jparam{session no}\}} and \jcsmacro{AdjSessionBlockNumber}).
	\end{itemize}
\item 2018/12/14 (v. 1.7):
	\begin{itemize}
		\item Allow \joption{options} in \jcsmacro{makebeamerprogram} also with \joption{blocksonly}.
		\item Add \joption{overlayarg} option to \jcsmacro{makebeamerprogram}. See sec.~\ref{sec:csdata}.
		\item Add macros to receive session block titles (\jcsmacro{sesblocktitle},\\
		      \jcsmacro{SessionBlockTitle\{\jparam{session no}\}} and \jcsmacro{AdjSessionBlockTitle}).
	\end{itemize}
\item 2018/09/26 (v. 1.6):
	\begin{itemize}
	    \item Bugfix: Do not output session title-subtitle separator if there is no	subtitle.
	    \item Add \joption{blocksonly} and \joption{uncover} options to \jcsmacro{makebeamerprogram}.
	    \item Add \joption{draft} option to \jcsmacro{NewSession}.
    \end{itemize}
\item 2017/03/12 (v. 1.5):
	\begin{itemize}
		\item Add \joption{english} style file (proof of concept).
		\item Add \joption{datakeys} option to \jsmt.
		\item Add package option \joption{sestitlesep} and macro \jcsmacro{SessionTitleSep} that
		allow for the definition of a global session title-subtitle	separator.
	\end{itemize}
\item 2016/10/19 (v. 1.4):
	\begin{itemize}
		\item Add instructor short forms for lecture and session.
		\item Add starred versions of the \jcsmacro{sesinstructor}, \jcsmacro{SessionInstructor} and \jcsmacro{AdjSessionInstructor} macros that only produce output if the session instructor differs from the general course instructor.
		\item All short\{instructor|title\} macros now output the long version if no short form was defined.
		\item Add boolean tests for unique short forms and session instructors.
	\end{itemize}
\item 2016/08/31 (v. 1.3):
     \begin{itemize}
     	\item \textbf{Backwards-incompatible change!} Use \textsf{datetime2} instead of \textsf{datetime} and \textsf{advdate}.
     	      This changes the interface to set the lecture date format. See sec.~\ref{sec:custdate}.
     	\item Fix output of \jcsmacro{SetBreaks} when no \joption{span} argument is given.
     	\item Protect content of some \jcsmacro{Lec}* macros and options to allow for commas in
     	      titles etc.
        \item Microtypographic improvement in default definition of the lecture date format.
	 \end{itemize}
\item 2016/05/11 (v. 1.2):
    \begin{itemize}	
	\item Fix line breaking with empty session titles before course breaks. 
	\item New commands for getting adjacent session data:
	   \begin{itemize}	
		  \item \jcsmacro{AdjSessionFullTitle}
		  \item \jcsmacro{AdjSessionDate}
		  \item \jcsmacro{AdjSessionShortTitle}
		  \item \jcsmacro{AdjSessionTitle}
		  \item \jcsmacro{AdjSessionShortDate}
		  \item \jcsmacro{AdjSessionRoom}
		  \item \jcsmacro{AdjSessionInstructor}\enlargethispage{\baselineskip}
		  \item \jcsmacro{AdjSessionPresStudents}
	  \end{itemize}
    \end{itemize}
\item 2015/07/15 (v. 1.1):
     \begin{itemize}
       \item Include and document the \jsmt\ package.
       \item Add support for student presentation lists. This introduces the new macros
             \jcsmacro{makepreslist}, \jcsmacro{makepreslist*},
             \jcsmacro{makebeamerpreslist}, \jcsmacro{setfirstpressession} and
             \jcsmacro{setlastpressession}.
       \item Major revision of the manual.
     \end{itemize}
\item 2015/07/07 (v. 1.0.2):
     \begin{itemize}
       \item Permit \jcsmacro{LecType} and \jcsmacro{LecYear} in the document body again
             (fix a regression introduced by v. 1.0).
     \end{itemize}
\item 2015/02/27 (v.\,1.0.1):
     \begin{itemize}
       \item Properly expand \jcsmacro{leccancel}
             (fix a regression introduced by v. 1.0).
     \end{itemize}
\item 2015/02/27 (v.\,1.0):
     \begin{itemize}
       \item Lecture types and term types are not hardcoded anymore and can be freely
             modified and extended via \jcsmacro{DefLecType} and \jcsmacro{DefSemType}.
       \item Add option to add the instructor's name to both the course (via macro
             \jcsmacro{LecInstructor} or \joption{instructor} package option) and
             to specific sessions (via \joption{instructor} option of
             \jcsmacro{NewSession}).
       \item Add option to add names of presenting students to a session (via option
             \joption{presstudents}).
       \item Add option to record the office number (\jcsmacro{SetOfficeNumber}
             or package option \joption{officenumber}).
       \item Fix expansion issue in \joption{options} option of \jcsmacro{makebeamerprogram}.
       \item Add option to automatically shift all subsequent sessions by a certain value
             after every \emph{n}th session (\jcsmacro{SetAutoOffset}).
       \item Add \joption{cologne} style file (suggested by Dominik Waßenhoven).
    \end{itemize}
\item 2015/02/08 (v. 0.9):
     \begin{itemize}
       \item Add a way to adjust the interval between two sessions
             (\joption{interval} package option and \jcsmacro{LecInterval} macro).
       \item Add a way to shift session dates from the regular interval
             (by means of macro \jcsmacro{SetLecOffset}).
       \item Add option and macro to set a room for the course as well as
             a session-specific room.
       \item Rename option \joption{weeks} of the \jcsmacro{SetBreaks} macro
             to \joption{units} (\joption{weeks} still works, but is marked deprecated).
    \end{itemize} 
\item 2015/01/17 (v.\,0.8):
    \begin{itemize}
       \item Fix corruption of \jfcsmacro{today}.
       \item Add framework for different styles.
       \item Add option and macro to set a room for the course as well as
             a session-specific room.
       \item Remove redundant lecture type \joption{vo} from the documentation.
       \item Change \jcsmacro{leccancel} default definition in favour of better
             translatability.
    \end{itemize}\filbreak
\item 2015/01/04 (v.\,0.7):
    \begin{itemize}
       \item Add macro for resetting \jfcsmacro{today}.
       \item Minor corrections of the manual.
    \end{itemize}
\item 2014/12/12 (v.\,0.6\,b): Minor corrections of the manual and example files.
\item 2014/12/12 (v.\,0.6): Initial release to CTAN.
\end{itemize}

\end{document}
